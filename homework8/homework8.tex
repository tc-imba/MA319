\documentclass[11pt,a4paper]{article}
\usepackage{../ma319}
\semester{Fall}
\year{2019}
\subtitlenumber{8}
\author{刘逸灏 (515370910207)}

\begin{document}

\maketitle

\section{2.3/5}
\begin{problem}
求解热传导方程(3.17)的柯西问题, 已知
\begin{enumerate}
  \item $u|_{t=0}=\sin x$,
  \item 用延拓法求解半有界直线上的热传导方程(3.17), 假设
        $$\left\{\begin{aligned}
             & u(x,0)=\varphi(x)\quad(0<x<\infty), \\
             & u(0,t)=0.
          \end{aligned}\right.$$
\end{enumerate}
\end{problem}

\subsection*{(i)}
初值条件为
$$\varphi(x)=u|_{t=0}=\sin x.$$
故
\begin{align*}
  u(x,t)
   & =\frac{1}{2a\sqrt{\pi t}}\int_{-\infty}^\infty\varphi(\xi)e^{-\frac{(x-\xi)^2}{4a^2t}}d\xi                                                               \\
   & =\frac{1}{2a\sqrt{\pi t}}\int_{-\infty}^\infty\varphi(x+2a\sqrt{t}\eta)e^{-\eta^2}d(x+2a\sqrt{t}\eta)                                                    \\
   & =\frac{1}{\sqrt{\pi}}\int_{-\infty}^\infty\sin(x+2a\sqrt{t}\eta)e^{-\eta^2}d\eta                                                                         \\
   & =\frac{1}{\sqrt{\pi}}\int_{-\infty}^\infty\left[\sin x\cos(2a\sqrt{t}\eta)+\cos x\sin(2a\sqrt{t}\eta)\right]e^{-\eta^2}d\eta                             \\
   & =\frac{\sin x}{\sqrt{\pi}}\int_{-\infty}^\infty\cos(2a\sqrt{t}\eta)e^{-\eta^2}d\eta                                                                      \\
   & =\frac{\sin x}{2\sqrt{\pi}}\int_{-\infty}^\infty \left[e^{-\eta^2+i2a\sqrt{t}\eta}+e^{-\eta^2-i2a\sqrt{t}\eta}\right]d\eta                               \\
   & =\frac{\sin x}{2\sqrt{\pi}}e^{-a^2t}\left[\int_{-\infty}^\infty e^{-(\eta-ia\sqrt{t})^2}d\eta+\int_{-\infty}^\infty e^{-(\eta+ia\sqrt{t})^2}d\eta\right] \\
   & =\frac{\sin x}{\sqrt{\pi}}e^{-a^2t}\int_{-\infty}^\infty e^{-\chi^2}d\chi                                                                                \\
   & =e^{-a^2t}\sin x.
\end{align*}

\subsection*{(2)}

使用奇延拓
$$u(x,0)=\left\{\begin{aligned}
     & \varphi(x)   & \quad(0<x<\infty),  \\
     & 0            & \quad (x=0),        \\
     & -\varphi(-x) & \quad(-\infty<x<0).
  \end{aligned}\right.$$
代入公式可得
\begin{align*}
  u(x,t)
   & =\frac{1}{2a\sqrt{\pi t}}\int_{-\infty}^\infty u(\xi,0)e^{-\frac{(x-\xi)^2}{4a^2t}}d\xi                   \\
   & =\frac{1}{2a\sqrt{\pi t}}\left(\int_0^\infty\varphi(\xi)e^{-\frac{(x-\xi)^2}{4a^2t}}d\xi
  +\int_{-\infty}^0-\varphi(-\xi)e^{-\frac{(x-\xi)^2}{4a^2t}}d\xi\right)                                       \\
   & =\frac{1}{2a\sqrt{\pi t}}\left(\int_0^\infty\varphi(\xi)e^{-\frac{(x-\xi)^2}{4a^2t}}d\xi
  +\int_0^\infty-\varphi(\xi)e^{-\frac{(x+\xi)^2}{4a^2t}}d\xi\right)                                           \\
   & =\frac{1}{2a\sqrt{\pi t}}\int_0^\infty\varphi(\xi)e^{-\frac{x^2+\xi^2}{4a^2t}}
  \left(e^{\frac{2x\xi}{4a^2t}}-e^{-\frac{2x\xi}{4a^2t}}\right)d\xi                                            \\
   & =\frac{1}{a\sqrt{\pi t}}\int_0^\infty\varphi(\xi)e^{-\frac{x^2+\xi^2}{4a^2t}}\sinh\frac{x\xi}{2a^2t}d\xi.
\end{align*}

\section{2.3/7}
\begin{problem}
证明: 如果$u_1(x,t)$, $u_2(x,t)$分别是下述两个定解问题的解:
$$\left\{\begin{aligned}
     & \frac{\partial u_1}{\partial t}=a^2\frac{\partial^2u_1}{\partial x^2}, \\
     & u_1|_{t=0}=\varphi_1(x);
  \end{aligned}\qquad\right\{\begin{aligned}
     & \frac{\partial u_2}{\partial t}=a^2\frac{\partial^2u_2}{\partial y^2}, \\
     & u_2|_{t=0}=\varphi_2(y).
  \end{aligned}$$
则$u(x,y,t)=u_1(x,t)u_2(y,t)$是定解问题
$$\left\{\begin{aligned}
     & \frac{\partial u}{\partial t}=a^2\left(\frac{\partial^2u}{\partial x^2}+\frac{\partial^2u}{\partial y^2}\right), \\
     & u|_{t=0}=\varphi_1(x)\varphi_2(y)
  \end{aligned}\right.$$
的解.
\end{problem}

由$u(x,y,t)=u_1(x,t)u_2(y,t)$可得
$$\frac{\partial u}{\partial t}=\frac{\partial u_1}{\partial t}u_2+\frac{\partial u_2}{\partial t}u_1=a^2\left(\frac{\partial^2u_1}{\partial x^2}u_2+\frac{\partial^2u_2}{\partial y^2}u_1\right)=a^2\left(\frac{\partial^2u_1u_2}{\partial x^2}+\frac{\partial^2u_1u_2}{\partial y^2}\right)=a^2\left(\frac{\partial^2u}{\partial x^2}+\frac{\partial^2u}{\partial y^2}\right),$$
$$u|_{t=0}=u_1|_{t=0}\cdot u_2|_{t=0}=\varphi_1(x)\varphi_2(y).$$
故得证.

\section{2.3/8}
\begin{problem}
导出下列热传导方程柯西问题的解的表达式:
$$\left\{\begin{aligned}
     & \frac{\partial u}{\partial t}=a^2\left(\frac{\partial^2u}{\partial x^2}+\frac{\partial^2u}{\partial y^2}\right), \\
     & u|_{t=0}=\sum_{i=0}^n\alpha_i(x)\beta_i(y).
  \end{aligned}\right.$$
\end{problem}

由上题结论易知
$$\left\{\begin{aligned}
     & \frac{\partial u_i}{\partial t}=a^2\left(\frac{\partial^2u_i}{\partial x^2}+\frac{\partial^2u_i}{\partial y^2}\right), \\
     & u_i|_{t=0}=\alpha_i(x)\beta_i(y).
  \end{aligned}\right.$$
的解$u_i(x,y,t)$为
\begin{align*}
  u_i(x,y,t)
   & =u_{i1}(x,t)y_{i2}(y,t)                                                                                                                    \\
   & =\frac{1}{2a\sqrt{\pi}t}\int_{-\infty}^\infty\alpha_i(\xi)e^{-\frac{(x-\xi)^2}{4a^2t}}d\xi
  \cdot\frac{1}{2a\sqrt{\pi}t}\int_{-\infty}^\infty\beta_i(\eta)e^{-\frac{(y-\eta)^2}{4a^2t}}d\eta                                              \\
   & =\frac{1}{4\pi a^2t}\int_{-\infty}^\infty\int_{-\infty}^\infty\alpha_i(\xi)\beta_i(\eta)e^{-\frac{(x-\xi)^2+(y-\eta)^2}{4a^2t}}d\xi d\eta.
\end{align*}

根据叠加原理
$$u(x,y,t)=\sum_{i=0}^n u_i(x,y,t)=\sum_{i=0}^n\frac{1}{4\pi a^2t}\int_{-\infty}^\infty\int_{-\infty}^\infty\alpha_i(\xi)\beta_i(\eta)e^{-\frac{(x-\xi)^2+(y-\eta)^2}{4a^2t}}d\xi d\eta.$$

\section{2.4/2}
\begin{problem}
利用证明热传导方程极值原理的方法, 证明满足方程$\dfrac{\partial^2u}{\partial x^2}+\dfrac{\partial^2u}{\partial y^2}=0$的函数在有界闭区域上的最大值不会超过它在边界上的最大值.
\end{problem}

设$u(x,y)$在区域$\Omega$内连续, 并且在区域内部满足方程$\dfrac{\partial^2u}{\partial x^2}+\dfrac{\partial^2u}{\partial y^2}=0$, 且$\Gamma$为区域$\Omega$的边界. 设$M$为$\Omega$内$u(x,y)$的最大值, $m$为$\Gamma$上$u(x,y)$的最大值. 使用反证法, 如果命题不真, 那么$M>m$. 此时在$\Omega$内一定存在着一点$(x^*,y^*)$, 使函数$u(x,y)$在该点取值$M$. 作函数
$$V(x,y)=u(x,y)+\frac{M-m}{4R^2}[(x-x^*)^2+(y-y^*)^2],$$
其中$R^2>(x-x^*)^2+(y-y^*)^2$. 由于在$\Gamma$上
$$V(x,y)<m+\frac{M-m}{4}=\frac{M}{4}+\frac{3}{4}m=\theta M\quad (0<\theta<1),$$
而$$V(x^*,y^*)=M,$$
因此, 函数$V(x,y)$和$u(x,y)$一样, 它不在$\Gamma$上取到最大值. 设$V(x,y)$在$\Omega$中的某一点$(x_1,y_1)$上取到最大值, 则在此点应有$\dfrac{\partial^2 V}{\partial x^2}\leqslant0$, $\dfrac{\partial^2 V}{\partial y^2}\leqslant0$, 因此在点$(x_1,y_1)$处
$$\frac{\partial^2 V}{\partial x^2}+\frac{\partial^2 V}{\partial y^2}\leqslant 0.$$
但由直接计算方程可得
$$\frac{\partial^2 V}{\partial x^2}+\frac{\partial^2 V}{\partial y^2}=\frac{\partial^2 u}{\partial x^2}+\frac{M-m}{2R^2}+\frac{\partial^2 u}{\partial y^2}+\frac{M-m}{2R^2}=\frac{\partial^2 u}{\partial x^2}+\frac{\partial^2 u}{\partial y^2}+\frac{M-m}{R^2}>0,$$
这就得到矛盾. 这说明原先的假设时不正确的, 证毕.

\section{2.4/3}
\begin{problem}
导出初边值问题
$$\left\{\begin{aligned}
     & u_t-a^2u_{xx}=f(x,t),                                                                                        \\
     & u|_{x=0}=\mu_1(x),\quad \left.\left(\frac{\partial u}{\partial x}+hu\right)\right|_{x=l}=\mu_2(t)\quad (h>0) \\
     & u|_{t=0}=\varphi(x)
  \end{aligned}\right.$$
的解$u(x,t)$在$R_T:\{0\leqslant t\leqslant T,0\leqslant x\leqslant l\}$中满足的估计.
$$u(x,t)\leqslant e^{\lambda T}\max\left(0,\max_{0\leqslant x\leqslant l}\varphi(x),\max_{0\leqslant t\leqslant T}\left(e^{-\lambda t}\mu_1(t),\frac{e^{-\lambda r}\mu_2(t)}{h}\right),\frac{1}{\lambda}\max_{R_T}(e^{-\lambda t}f)\right),$$
其中$\lambda>0$为任意正常数.
\end{problem}

\end{document}
