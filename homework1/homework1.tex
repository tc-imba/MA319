\documentclass[11pt,a4paper]{article}
\usepackage{../ma319}
\semester{Fall}
\year{2019}
\subtitlenumber{1}
\author{刘逸灏 (515370910207)}

\begin{document}
\maketitle

\section{1.1/6}

对于$F(\xi)=F(x-at)$
$$\xi=x-at,\quad \frac{\partial \xi}{\partial t}=-a,\quad \frac{\partial \xi}{\partial x}=1,$$
$$\frac{\partial F}{\partial t}=\frac{\partial F}{\partial \xi}\cdot\frac{\partial \xi}{\partial t}=-aF'(\xi),$$
$$\frac{\partial^2 F}{\partial t^2}=\frac{\partial}{\partial t}[-aF'(\xi)]=\frac{\partial}{\partial \xi}[-aF'(\xi)]\cdot\frac{\partial \xi}{\partial t}=a^2F''(\xi),$$
$$\frac{\partial F}{\partial x}=\frac{\partial F}{\partial \xi}\cdot\frac{\partial \xi}{\partial x}=F'(\xi),$$
$$\frac{\partial^2 F}{\partial x^2}=\frac{\partial}{\partial x}[F'(\xi)]=\frac{\partial}{\partial \xi}[F'(\xi)]\cdot\frac{\partial \xi}{\partial x}=F''(\xi).$$

故$$\frac{\partial^2 F}{\partial t^2}-a^2\frac{\partial^2 F}{\partial x^2}=0.$$

对于$G(\xi)=G(x+at)$
$$\xi=x+at,\quad \frac{\partial \xi}{\partial t}=a,\quad \frac{\partial \xi}{\partial x}=1,$$
$$\frac{\partial G}{\partial t}=\frac{\partial G}{\partial \xi}\cdot\frac{\partial \xi}{\partial t}=aG'(\xi),$$
$$\frac{\partial^2 G}{\partial t^2}=\frac{\partial}{\partial t}[-aG'(\xi)]=\frac{\partial}{\partial \xi}[aG'(\xi)]\cdot\frac{\partial \xi}{\partial t}=a^2G''(\xi),$$
$$\frac{\partial G}{\partial x}=\frac{\partial G}{\partial \xi}\cdot\frac{\partial \xi}{\partial x}=G'(\xi),$$
$$\frac{\partial^2 G}{\partial x^2}=\frac{\partial}{\partial x}[G'(\xi)]=\frac{\partial}{\partial \xi}[G'(\xi)]\cdot\frac{\partial \xi}{\partial x}=G''(\xi).$$

故$$\frac{\partial^2 G}{\partial t^2}-a^2\frac{\partial^2 G}{\partial x^2}=0.$$

\section{1.1/7}

$$u(x,y,t)=(t^2-x^2-y^2)^{1/2},\quad t^2-x^2-y^2>0.$$

一阶偏导数为
$$\frac{\partial u}{\partial t}=\frac{1}{2}(t^2-x^2-y^2)^{-1/2}\cdot 2t=\frac{t}{u},$$
$$\frac{\partial u}{\partial x}=\frac{1}{2}(t^2-x^2-y^2)^{-1/2}\cdot -2x=-\frac{x}{u},$$
$$\frac{\partial u}{\partial y}=\frac{1}{2}(t^2-x^2-y^2)^{-1/2}\cdot -2y=-\frac{y}{u}.$$

二阶偏导数为
$$\frac{\partial^2 u}{\partial t^2}=\frac{\partial}{\partial t}\frac{t}{u}=\frac{\partial t}{\partial t}\cdot \frac{1}{u}+\frac{\partial}{\partial t}\frac{1}{u}\cdot t=\frac{1}{u}-\frac{3t^2}{u^3},$$
$$\frac{\partial^2 u}{\partial x^2}=-\frac{\partial}{\partial x}\frac{x}{u}=-\frac{\partial x}{\partial x}\cdot \frac{1}{u}-\frac{\partial}{\partial x}\frac{1}{u}\cdot t=-\frac{1}{u}-\frac{3x^2}{u^3},$$
$$\frac{\partial^2 u}{\partial y^2}=-\frac{\partial}{\partial y}\frac{y}{u}=-\frac{\partial y}{\partial y}\cdot \frac{1}{u}-\frac{\partial}{\partial y}\frac{1}{u}\cdot t=-\frac{1}{u}-\frac{3y^2}{u^3}.$$

故
$$\frac{\partial^2 u}{\partial t^2}-\frac{\partial^2 u}{\partial x^2}-\frac{\partial^2 u}{\partial y^2}=\frac{3}{u}+\frac{-3t^2+3x^2+3y^2}{u^3}=\frac{3}{u}+\frac{-3u^2}{u^3}=0,$$
$$\frac{\partial^2 u}{\partial t^2}=\frac{\partial^2 u}{\partial x^2}+\frac{\partial^2 u}{\partial y^2}.$$

\section{1.2/1}

$$\left(1-\frac{x}{h}\right)^2\frac{\partial^2 u}{\partial t^2}=a^2\frac{\partial}{\partial x}\left[\left(1-\frac{x}{h}\right)^2\frac{\partial u}{\partial x}\right],$$
$$(h-x)^2\frac{\partial^2 u}{\partial t^2}=a^2\frac{\partial}{\partial x}\left[(h-x)^2\frac{\partial u}{\partial x}\right].$$

设$u(x,t)=v(x,t)/(h-x)$
$$\frac{\partial^2 u}{\partial t^2}=\frac{1}{h-x}\frac{\partial^2 v}{\partial t^2},\quad (h-x)^2\frac{\partial^2 u}{\partial t^2}=(h-x)\frac{\partial^2 v}{\partial t^2}.$$
$$\frac{\partial u}{\partial x}=\frac{\partial v}{\partial x}\cdot\frac{1}{h-x}+\frac{\partial}{\partial x}\left(\frac{1}{h-x}\right)\cdot v=\frac{1}{h-x}\frac{\partial v}{\partial x}+\frac{v}{(h-x)^2},$$
$$\frac{\partial}{\partial x}\left[(h-x)^2\frac{\partial u}{\partial x}\right]=\frac{\partial}{\partial x}\left[(h-x)\frac{\partial v}{\partial x}+v\right]=
  \frac{\partial^2 v}{\partial x^2}\cdot(h-x)+\frac{\partial}{\partial x}(h-x)\cdot\frac{\partial v}{\partial x}+\frac{\partial v}{\partial x}=(h-x)\frac{\partial^2 v}{\partial x^2}.$$

故
$$\frac{\partial^2 v}{\partial t^2}-a^2\frac{\partial^2 v}{\partial x^2}=0.$$

其通解可以写为
$$v(x,t)=F(x-at)+G(x+at),$$
$$u(x,t)=\frac{v(x,t)}{h-x}=\frac{F(x-at)+G(x+at)}{h-x}.$$

且满足初值条件
$$t=0:\quad v=(h-x)\varphi(x),\quad \frac{\partial v}{\partial t}=(h-x)\psi(x).$$

代入达朗贝尔公式得
$$v(x,t)=\frac{(h-x+at)\varphi(x-at)+(h-x-at)\varphi(x+at)}{2}+\frac{1}{2a}\int_{x-at}^{x+at}(h-\alpha)\psi(\alpha)d\alpha,$$
$$u(x,t)=\frac{(h-x+at)\varphi(x-at)+(h-x-at)\varphi(x+at)}{2(h-x)}+\frac{1}{2a(h-x)}\int_{x-at}^{x+at}(h-\alpha)\psi(\alpha)d\alpha.$$

\section{1.2/2}

齐次波动方程初值问题的解仅由右传播波组成时
$$G(x+at)=C_1,$$
$$G(x)=\frac{1}{2}\varphi(x)+\frac{1}{2a}\int_{x_0}^x\psi(\alpha)d\alpha-\frac{C}{2a}=C_1.$$

故初始条件$\varphi(x)$, $\psi(x)$需要满足
$$\varphi(x)+\frac{1}{a}\int_{x_0}^x\psi(\alpha)d\alpha=C_2.$$
其中$C$, $C_1$, $C_2$为常数.

\section*{例题}

$$\left\{\begin{aligned}
     & \frac{\partial^2u}{\partial x\partial y}=0,\quad |x|<1,\quad |y|<1 \\
     & u\mid_{y=x^2}=\varphi(x)                                           \\
     & u_y\mid_{y=x^2}=\psi(x)
  \end{aligned}\right..$$

方程的通解为
$$u(x,y)=F(x)+G(y)+C,$$

代入初值条件得
$$\varphi(x)=u|_{y=x^2}=F(x)+G(x^2)+C,$$
$$\psi(x)=u_y|_{y=x^2}=G'(x^2).$$

故
$$G'(x)=\psi(\sqrt{|x|}),$$
$$G(x)=\int_{x_0}^x\psi(\sqrt{|\xi|})d\xi+C_1,$$
$$F(x)=\varphi(x)-G(x^2)-C=\varphi(x)-\int_{x_0}^{x^2}\psi(\sqrt{|\xi|})d\xi-C_1-C,$$
$$u(x,y)=F(x)+G(y)+C=\varphi(x)+\int_{x^2}^{y}\psi(\sqrt{|\xi|})d\xi.$$

\end{document}
