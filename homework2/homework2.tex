\documentclass[11pt,a4paper]{article}
\usepackage{../ma319}
\semester{Fall}
\year{2019}
\subtitlenumber{2}
\author{刘逸灏 (515370910207)}

\begin{document}
\maketitle

\section{1.2/3}
\begin{problem}
利用传播波法, 求解波动方程的古尔萨问题:
$$\left\{\begin{aligned}
     & \frac{\partial^2 u}{\partial t^2}=a^2\frac{\partial^2u}{\partial x^2}, \\
     & u|_{x-at=0}=\varphi(x),                                                \\
     & u|_{x+at=0}=\psi(x),\quad \varphi(0)=\psi(0).
  \end{aligned}\right.$$
\end{problem}

方程的通解为
$$u(x,t)=F(x-at)+G(x+at),$$

代入初值条件得
$$\varphi(x)=u|_{x-at=0}=F(0)+G(2x),$$
$$\psi(x)=u|_{x+at=0}=F(2x)+G(0).$$

故
$$F(x)=\psi(x/2)-G(0),$$
$$G(x)=\varphi(x/2)-F(0),$$
$$\varphi(0)=\psi(0)=F(0)+G(0).$$

$$u(x,t)=F(x-at)+G(x+at)=\psi\left(\frac{x-at}{2}\right)+\varphi\left(\frac{x+at}{2}\right)-\varphi(0).$$

\section{1.2/4}
\begin{problem}
对非齐次波动方程的初值问题(2.5), (2.6), 证明: 当$f(x,t)$不变时,
\begin{enumerate}
  \item 如果初始条件在$x$轴的区间$[x_1,x_2]$发生变化, 那么对应的解在区间$[x_1,x_2]$的影响区域以外不发生变化;
  \item 在$x$轴区间$[x_1,x_2]$上所给的初始条件唯一地确定区间$[x_1,x_2]$的决定区域中解的数值.
\end{enumerate}
\end{problem}

非齐次初值问题的解为
$$u(x,t)=\frac{\varphi(x-at)+\varphi(x+at)}{2}+\frac{1}{2a}\int_{x-at}^{x+at}\psi(\alpha)d\alpha+\frac{1}{2a}\int_0^t\int_{x-a(t-\tau)}^{x+a(t-\tau)}f(\xi,\tau)d\xi d\tau.$$

其中与$\varphi(x)$和$\psi(x)$有关的定义域范围都是$[x-at,x+at]$.

\subsection*{(1)}
区间$[x_1,x_2]$的影响区域为$$x_1-at\leqslant x\leqslant x_2+at,$$

不受影响的区域为$$x\leqslant x_1-at\quad\text{和}\quad x\geqslant x_2+at,$$

对应的$\varphi(x)$和$\psi(x)$定义域范围是$$x\leqslant x_1\quad\text{和}\quad x\geqslant x_2.$$

故当$\varphi(x)$和$\psi(x)$在$[x_1,x_2]$上变化时, 以上定义域内函数取值不变, 对应解也不变.

\subsection*{(2)}

区间$[x_1,x_2]$的决定区域为$$x_1+at\leqslant x\leqslant x_2-at,$$

对应的$\varphi(x)$和$\psi(x)$定义域范围是$$x_1\leqslant x\leqslant x_2.$$

故$[x_1,x_2]$上所给的初始条件唯一地确定该区间解的数值.

\section{1.2/5}
\begin{problem}
求解
$$\left\{\begin{aligned}
     & u_{tt}-a^2u_{xx}=0,\quad x>0,t>0,       \\
     & u|_{t=0}=\varphi(x),\quad u_t|_{t=0}=0, \\
     & u_x-ku_t|_{x=0}=0,
  \end{aligned}\right. $$
其中$k$为正常数.
\end{problem}

方程的通解为
$$u(x,t)=F(x-at)+G(x+at),$$

代入初值条件得
$$\varphi(x)=u|_{t=0}=F(x)+G(x),$$
$$0=u_t|_{t=0}=a[-F'(x)+G'(x)].$$
$$0=u_x-ku_t\mid_{x=0}=F'(-at)+G'(at)-ka[-F'(-at)+G'(at)].$$

故当$x\geqslant 0$时
$$F(x)-G(x)=C=F(0)-G(0),$$
$$F(x)=\frac{1}{2}\varphi(x)+\frac{C}{2}=\frac{1}{2}\varphi(x)+\frac{F(0)-G(0)}{2},$$
$$G(x)=\frac{1}{2}\varphi(x)-\frac{C}{2}=\frac{1}{2}\varphi(x)-\frac{F(0)-G(0)}{2}.$$

$$(ka+1)F'(-at)-(ka-1)G'(at)=0,$$
$$(ka+1)F'(-x)-(ka-1)G'(x)=0,$$
$$\int_0^x[(ka+1)F'(-\xi)-(ka-1)G'(\xi)]d\xi=0,$$
$$-(ka+1)F(-x)-(ka-1)G(x)=C_1=-(ka+1)F(0)-(ka-1)G(0),$$
$$F(-x)=\frac{(1-ka)G(x)-C_1}{1+ka}=\frac{1-ka}{1+ka}G(x)+F(0)+\frac{ka-1}{1+ka}G(0).$$

当$x-at\geqslant 0$时
$$u(x,t)=F(x-at)+G(x+at)=\frac{\varphi(x-at)+\varphi(x+at)}{2}.$$

当$x-at<0$时
\begin{align*}
  u(x,t) & =F(-(at-x))+G(x+at)                                                                                                                                 \\
         & =\frac{1-ka}{1+ka}G(at-x)+F(0)+\frac{ka-1}{1+ka}G(0)+G(x+at)                                                                                        \\
         & =\frac{1-ka}{1+ka}\left[\frac{1}{2}\varphi(at-x)-\frac{F(0)-G(0)}{2}\right]+F(0)+\frac{ka-1}{1+ka}G(0)+\frac{1}{2}\varphi(x+at)-\frac{F(0)-G(0)}{2} \\
         & =\frac{1-ka}{2(1+ka)}\varphi(at-x)+\frac{1}{2}\varphi(x+at)+\frac{[-(1-ka)+(1+ka)]F(0)+[-(1-ka)+(1+ka)]G(0)}{2(1+ka)}                               \\
         & =\frac{1-ka}{2(1+ka)}\varphi(at-x)+\frac{1}{2}\varphi(x+at)+\frac{ka}{1+ka}\varphi(0).
\end{align*}

故
$$u(x,t)=\frac{1}{2}\varphi(x+at)+\left\{\begin{aligned}
     & \frac{1}{2}\varphi(x-at),                                    & x-at\geqslant 0 \\
     & \frac{1-ka}{2(1+ka)}\varphi(at-x)+\frac{ka}{1+ka}\varphi(0), & x-at<0
  \end{aligned}\right..$$

\section{1.2/6}
\begin{problem}
求解初边值问题
$$\left\{\begin{aligned}
     & u_{tt}-u_{xx}=0,\quad 0<t<kx,k>1,          \\
     & u|_{t=0}=\varphi_0(x),\quad x\geqslant0,   \\
     & u_t|_{t=0}=\varphi_1(x),\quad x\geqslant0, \\
     & u|_{t=kx}=\psi(x),
  \end{aligned}\right.$$
其中$\varphi_0(0)=\psi(0)$.
\end{problem}

方程的通解为
$$u(x,t)=F(x-t)+G(x+t),$$

代入初值条件得
$$\varphi_0(x)=u|_{t=0}=F(x)+G(x),$$
$$\varphi_1(x)=u_t|_{t=0}=a[-F'(x)+G'(x)].$$
$$\psi(x)=u|_{t=kx}=F((1-k)x)+G((1+k)x).$$

故当$x\geqslant 0$时
$$F(x)-G(x)=-\int_{x_0}^x\psi(\alpha)d\alpha+C,$$
$$F(x)=\frac{1}{2}\varphi_0(x)-\frac{1}{2}\int_{x_0}^x\varphi_1(\alpha)d\alpha+\frac{C}{2},$$
$$G(x)=\frac{1}{2}\varphi_0(x)+\frac{1}{2}\int_{x_0}^x\varphi_1(\alpha)d\alpha-\frac{C}{2}.$$

当$x-t\geqslant 0$时
$$u(x,t)=F(x-t)+G(x+t)=\frac{\varphi_0(x-t)+\varphi_0(x+t)}{2}+\frac{1}{2}\int_{x-t}^{x+t}\varphi_1(\alpha)d\alpha.$$

当$x-t\leqslant 0$时
$$u(x,x)=F(x-x)+G(x+x)=F(0)+G(2x)=\frac{\varphi_0(0)+\varphi_0(2x)}{2}+\frac{1}{2}\int_{0}^{2x}\varphi_1(\alpha)d\alpha,$$
$$G(x)=\frac{\varphi_0(0)+\varphi_0(x)}{2}+\frac{1}{2}\int_{0}^{x}\varphi_1(\alpha)d\alpha-F(0),$$
$$F((1-k)x)=\psi(x)-G((1+k)x)=\psi(x)-\frac{\varphi_0(0)+\varphi_0((1+k)x)}{2}-\frac{1}{2}\int_{0}^{(1+k)x}\varphi_1(\alpha)d\alpha+F(0),$$
$$F(x)=\psi\left(\frac{x}{1-k}\right)-\frac{1}{2}\varphi_0(0)-\frac{1}{2}\varphi_0\left(\frac{1+k}{1-k}x\right)-\frac{1}{2}\int_0^{\frac{1+k}{1-k}x}\varphi_1(\alpha)d\alpha+F(0).$$
\begin{align*}
  u(x,t) & =F(x-t)+G(x+t)                                                                                                                                                                     \\
         & =\psi\left(\frac{x-t}{1-k}\right)-\frac{1}{2}\varphi_0(0)-\frac{1}{2}\varphi_0\left(\frac{1+k}{1-k}(x-t)\right)-\frac{1}{2}\int_0^{\frac{1+k}{1-k}(x-t)}\varphi_1(\alpha)d\xi+F(0) \\
         & \quad+\frac{\varphi_0(0)+\varphi_0(x+t)}{2}+\frac{1}{2}\int_{0}^{x+t}\varphi_1(\xi)d\xi-F(0)                                                                                       \\
         & =\psi\left(\frac{x-t}{1-k}\right)-\frac{1}{2}\varphi_0\left(\frac{1+k}{1-k}(x-t)\right)+\frac{1}{2}\varphi_0(x+t)+\frac{1}{2}\int_{\frac{1+k}{1-k}(x-t)}^{x+t}\varphi_1(\xi)d\xi.
\end{align*}

故
$$u(x,t)=\left\{\begin{aligned}
     & \frac{\varphi_0(x-t)+\varphi_0(x+t)}{2}+\frac{1}{2}\int_{x-t}^{x+t}\varphi_1(\xi)d\xi,                                                                                           & 0\leqslant t\leqslant x \\
     & \psi\left(\frac{x-t}{1-k}\right)-\frac{1}{2}\varphi_0\left(\frac{1+k}{1-k}(x-t)\right)+\frac{1}{2}\varphi_0(x+t)+\frac{1}{2}\int_{\frac{1+k}{1-k}(x-t)}^{x+t}\varphi_1(\xi)d\xi, & x<t<kx
  \end{aligned}\right..$$

\section{1.2/7}
\begin{problem}
求边值问题
$$\left\{\begin{aligned}
     & u_{tt}-u_{xx}=0,\quad f(t)<x<t, \\
     & u|_{x=t}=\varphi(t),            \\
     & u|_{x=f(t)}=\psi(t)
  \end{aligned}\right.$$
的解, 其中$\varphi(0)=\psi(x)$, $x=f(t)$为由原点出发的, 介于$x=t$和$x=-t$之间的光滑曲线, 且$|f'(t)|\neq1$对一切$t$成立.
\end{problem}

方程的通解为
$$u(x,t)=F(x-t)+G(x+t),$$

代入初值条件得
$$\varphi(x)=u|_{t=x}=F(0)+G(2x),$$
$$\psi(x)=u|_{t=f(x)}=F(x-f(x))+G(x+f(x)).$$

解得
$$G(x)=\varphi(x/2)-F(0),$$
$$F(x-f(x))=\psi(x)-G(x+f(x))=\psi(x)-\varphi\left(\frac{x+f(x)}{2}\right)+F(0).$$

设$g(x)=x-f(x)$, 由于$f'(x)\neq 1$, 易知$g'(x)\neq 0$, 即$g(x)$为连续单调函数, $g^{-1}(x)$存在.
$$F(x-f(x))=F(g(x))=\psi(x)-\varphi\left(x-\frac{g(x)}{2}\right)+F(0),$$
$$F(x)=\psi(g^{-1}(x))-\varphi\left(g^{-1}(x)-\frac{x}{2}\right)+F(0).$$

故
\begin{align*}
  u(x,t) & =F(x-t)+G(x+t)                                                                                               \\
         & =\psi(g^{-1}(x-t))-\varphi\left(g^{-1}(x-t)-\frac{x-t}{2}\right)+F(0)+\varphi\left(\frac{x+t}{2}\right)-F(0) \\
         & =\psi(g^{-1}(x-t))+\varphi\left(g^{-1}(x-t)-\frac{x-t}{2}\right)+\varphi\left(\frac{x+t}{2}\right).
\end{align*}

\section{1.2/8}
\begin{problem}
求解波动方程的初值问题
$$\left\{\begin{aligned}
     & \frac{\partial^2u}{\partial t^2}-\frac{\partial^2u}{\partial x^2}=t\sin x, \\
     & u|_{t=0}=0,\quad \left.\frac{\partial u}{\partial t}\right|_{t=0}=\sin x.
  \end{aligned}\right.$$
\end{problem}

代入Kirchhoff公式得
\begin{align*}
  u(x,t) & =\frac{1}{2}\int_{x-t}^{x+t}\sin\xi d\xi+\frac{1}{2}\int_0^t\int_{x-(t-\tau)}^{x+(t-\tau)}\tau\sin\xi d\xi d\tau  \\
         & =\frac{1}{2}[\cos(x-t)-\cos(x+t)]+\frac{1}{2}\int_0^t\tau[\cos(x-t+\tau)-\cos(x+t-\tau)]d\tau                     \\
         & =\frac{1}{2}[\cos(x-t)-\cos(x+t)]+\frac{1}{2}[\cos(-x)-t\sin(-x)-\cos(t-x)]-\frac{1}{2}[\cos x-t\sin x-\cos(t+x)] \\
         & =t\sin x.
\end{align*}

\section{1.2/9}
\begin{problem}
求解波动方程的初值问题
$$\left\{\begin{aligned}
     & u_{tt}=a^2u_{xx}+\frac{tx}{(1+x^2)^2}, \\
     & u|_{t=0}=0,                            \\
     & u_t|_{t=0}=\frac{1}{1+x^2}.
  \end{aligned}\right.$$
\end{problem}

代入Kirchhoff公式得
\begin{align*}
  u(x,t) & =\frac{1}{2a}\int_{x-at}^{x+at}\frac{1}{1+\xi^2}d\xi+\frac{1}{2a}\int_0^t\int_{x-a(t-\tau)}^{x+a(t-\tau)}\frac{\tau\xi}{(1+\xi^2)^2}d\xi d\tau   \\
         & =\frac{1}{2a}[\arctan(x+at)-\arctan(x-at)]+\frac{1}{4a}\int_0^t\tau\left[\frac{1}{1+[x-a(t-\tau)]^2}-\frac{1}{1+[x+a(t-\tau)]^2}\right]d\tau     \\
         & =\frac{1}{2a}[\arctan(x+at)-\arctan(x-at)]+\frac{1}{4a^3}\left[(at-x)[\arctan x-\arctan(x-at)]+\frac{1}{2}\ln\frac{1+x^2}{1+(x-at)^2}\right]     \\
         & \quad-\frac{1}{4a^3}\left[-(at+x)[\arctan x-\arctan(x+at)]+\frac{1}{2}\ln\frac{1+x^2}{1+(x+at)^2}\right]                                         \\
         & =\frac{1}{2a}[\arctan(x+at)-\arctan(x-at)]+\frac{1}{8a^3}\ln\frac{1+(x+at)^2}{1+(x-at)^2}                                                        \\
         & \quad+\frac{1}{4a^3}[2at\arctan x+(x-at)\arctan(x-at)-(x+at)\arctan(x+at)]                                                                       \\
         & =\frac{t}{2a^2}\arctan x+\frac{x-at-2a^2}{4a^3}\arctan(x-at)-\frac{x+at-2a^2}{4a^3}\arctan(x+at)+\frac{1}{8a^3}\ln\frac{1+(x+at)^2}{1+(x-at)^2}.
\end{align*}

\end{document}
