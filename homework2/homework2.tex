\documentclass[11pt,a4paper]{article}
\usepackage{../ma319}
\semester{Fall}
\year{2019}
\subtitlenumber{2}
\author{刘逸灏 (515370910207)}

\begin{document}
\maketitle

\section{1.2/3}

方程的通解为
$$u(x,t)=F(x-at)+G(x+at),$$

代入初值条件得
$$\varphi(x)=u|_{x-at=0}=F(0)+G(2x),$$
$$\psi(x)=u|_{x+at=0}=F(2x)+G(0).$$

故
$$F(x)=\psi(x/2)-G(0),$$
$$G(x)=\varphi(x/2)-F(0),$$
$$\varphi(0)=\psi(0)=F(0)+G(0).$$

$$u(x,t)=F(x-at)+G(x+at)=\psi\left(\frac{x-at}{2}\right)+\varphi\left(\frac{x+at}{2}\right)-\varphi(0).$$

\section{1.2/4}

非齐次初值问题的解为
$$u(x,t)=\frac{\varphi(x-at)+\varphi(x+at)}{2}+\frac{1}{2a}\int_{x-at}^{x+at}\psi(\alpha)d\alpha+\frac{1}{2a}\int_0^t\int_{x-a(t-\tau)}^{x+a(t-\tau)}f(\xi,\tau)d\xi d\tau.$$

其中与$\varphi(x)$和$\psi(x)$有关的定义域范围都是$[x-at,x+at]$.

\subsection*{(1)}
区间$[x_1,x_2]$的影响区域为$$x_1-at\leqslant x\leqslant x_2+at,$$

不受影响的区域为$$x\leqslant x_1-at\quad\text{和}\quad x\geqslant x_2+at,$$

对应的$\varphi(x)$和$\psi(x)$定义域范围是$$x\leqslant x_1\quad\text{和}\quad x\geqslant x_2.$$

故当$\varphi(x)$和$\psi(x)$在$[x_1,x_2]$上变化时, 以上定义域内函数取值不变, 对应解也不变.

\subsection*{(2)}

区间$[x_1,x_2]$的决定区域为$$x_1+at\leqslant x\leqslant x_2-at,$$

对应的$\varphi(x)$和$\psi(x)$定义域范围是$$x_1\leqslant x\leqslant x_2.$$

故$[x_1,x_2]$上所给的初始条件唯一地确定该区间解的数值.

\section{1.2/5}
方程的通解为
$$u(x,t)=F(x-at)+G(x+at),$$

代入初值条件得
$$\varphi(x)=u|_{t=0}=F(x)+G(x),$$
$$0=u_t|_{t=0}=a[-F'(x)+G'(x)].$$
$$0=u_x-ku_t\mid_{x=0}=F'(x-at)+G'(x+at)-ka[-F'(-at)+G'(at)].$$

$$u(x,t)=\frac{\varphi(x-at)+\varphi(x+at)}{2}$$
$$u_x(x,t)=\frac{\varphi'(x-at)+\varphi'(x+at)}{2}$$
$$ku_t\mid_{x=0}=\frac{-ka\varphi'(-at)+ka\varphi'(at)}{2}$$

\end{document}
