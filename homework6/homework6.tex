\documentclass[11pt,a4paper]{article}
\usepackage{../ma319}
\semester{Fall}
\year{2019}
\subtitlenumber{6}
\author{刘逸灏 (515370910207)}

\begin{document}
\maketitle

\section{1.6/3}
\begin{problem}
证明波动方程
$$u_{tt}=a^2(u_{xx}+u_{yy})+f(x,y,t)$$
的自由项$f$在$L^2(K)$意义下作微小改变时, 对应的柯西问题的解$u$在$L^2(K)$意义下改变也是微小的, 其中$K$是由
$$(x-x_0)^2+(y-y_0)^2\leqslant(R-at)^2$$
所表示的锥体.
\end{problem}
在$K$内任一截面$\Omega_t$上成立能量不等式
$$E_0(\Omega_t)=\iint\limits_{\Omega_t}u^2dxdy,\quad E_1(\Omega_t)=\iint\limits_{\Omega_t}\left[u_t^2+a^2\left(u_x^2+u_y^2\right)\right]dxdy.$$
对$E_1(t)$有
\begin{align*}
  \frac{dE_1(\Omega_t)}{dt}
   & =\frac{d}{dt}\int_0^{R-at}\int_0^{2\pi r}\left[u_t^2+a^2\left(u_x^2+u_y^2\right)\right]dsdr                                        \\
   & =2\int_0^{R-at}\int_0^{2\pi r}\left[u_tu_{tt}+a^2\left(u_xu_{xt}+u_yu_{yt}\right)\right]dsdr-
  a\int_{\Gamma_t}\left[u_t^2+a^2\left(u_x^2+u_y^2\right)\right]ds.                                                                     \\
   & =2\int_0^{R-at}\int_0^{2\pi r}u_t\left[u_{tt}-a^2\left(u_x^2+u_y^2\right)\right]dsdr                                               \\
   & \quad+2\int_{\Gamma_t}\left\{a^2[u_xu_t\cos(n,x)+u_yu_t\sin(n,x)]-
  \frac{a}{2}\left[u_t^2+a^2\left(u_x^2+u_y^2\right)\right]\right\}ds.                                                                  \\
   & =2\iint\limits_{\Omega_t}u_tfdxdy-a\int_{\Gamma_t}\left[\left(au_x-u_t\cos(n,x)\right)^2+\left(au_y-u_t\cos(n,y)\right)^2\right]ds \\
   & \leqslant 2\iint\limits_{\Omega_t}u_tfdxdy\leqslant \iint\limits_{\Omega_t}u_t^2dxdy+\iint\limits_{\Omega_t}f^2dxdy
  \leqslant E_1(\Omega_t)+\iint\limits_{\Omega_t}f^2dxdy,
\end{align*}
$$\frac{d}{dt}e^{-t}E_1(\Omega_t)=-e^{-t}E_1(\Omega_t)+e^{-t}\frac{dE_1(\Omega_t)}{dt}\leqslant e^{-t}\iint\limits_{\Omega_t}f^2dxdy,$$
$$E_1(\Omega_t)\leqslant e^t\int_0^te^{-\tau}\int_0^l f^2dxd\tau+e^tE_1(\Omega_0)\leqslant e^tE_1(\Omega_0)+e^t\int_0^t\iint\limits_{\Omega_t}f^2dxdyd\tau=\overline{E_1(\Omega_t)}.$$
对$E_0(t)$有
$$\frac{dE_0(\Omega_t)}{dt}=2\iint\limits_{\Omega_t}uu_tdxdy-a\int_{\Gamma_t}u^2ds\leqslant2\iint\limits_{\Omega_t}uu_tdxdy\leqslant\iint\limits_{\Omega_t}u^2dxdy+\iint\limits_{\Omega_t}u_t^2dxdy\leqslant E_0(\Omega_t)+E_1(\Omega_t),$$
$$\frac{d}{dt}e^{-t}E_0(\Omega_t)=-e^{-t}E_0(\Omega_t)+e^{-t}\frac{dE_0(\Omega_t)}{dt}\leqslant e^{-t}E_1(\Omega_t),$$
$$E_0(\Omega_t)\leqslant e^t\int_0^t e^{-\tau}E_1(\Omega_\tau)d\tau+e^t E_0(\Omega_0)\leqslant e^t\overline{E_1(\Omega_t)}(1-{e^{-t}})+e^t E_0(\Omega_0)=e^tE_0(\Omega_0)+(e^t-1)\overline{E_1(\Omega_t)},$$
\begin{align*}
  E_0(\Omega_t)+E_1(\Omega_t)
   & \leqslant e^tE_0(\Omega_0)+(e^t-1)\overline{E_1(\Omega_t)}+E_1(\Omega_t)\leqslant e^t(E_0(\Omega_0)+\overline{E_1(\Omega_t)}) \\
   & =e^tE_0(\Omega_0)+e^{2t}E_1(\Omega_0)+e^{2t}\int_0^t\iint\limits_{\Omega_t}f^2dxdyd\tau.
\end{align*}
在$0\leqslant t\leqslant T$上
\begin{align*}
  E_0(\Omega_t)+E_1(\Omega_t)
   & \leqslant e^TE_0(\Omega_0)+e^{2T}E_1(\Omega_0)+e^{2T}\int_0^T\iint\limits_{\Omega_t}f^2dxdyd\tau      \\
   & \leqslant e^{2T}\left(E_0(\Omega_0)+E_1(\Omega_0)+\int_0^T\iint\limits_{\Omega_t}f^2dxdyd\tau\right).
\end{align*}
存在$C$使得
$$\sqrt{T(E_0(\Omega_t)+E_1(\Omega_t))}\leqslant \sqrt{Te^{2T}\left(E_0(\Omega_0)+E_1(\Omega_0)+\int_0^T\iint\limits_{\Omega_t}f^2dxdyd\tau\right)}\leqslant C\eta,$$
$$\sqrt{E_0(\Omega_0)+E_1(\Omega_0)+\int_0^T\iint\limits_{\Omega_t}f^2dxdyd\tau}\leqslant\eta.$$
任取$\varepsilon>0$, 可以找到$\eta=\dfrac{\varepsilon}{C}$, 使得
$$\|f_1-f_2\|_{L^2(K)}\leqslant \sqrt{E_0(\Omega_0)+E_1(\Omega_0)+\int_0^T\iint\limits_{\Omega_t}f^2dxdyd\tau}\leqslant \eta,$$
$$\|u_1-u_2\|_{L^2(K)}\leqslant \sqrt{T(E_0(\Omega_t)+E_1(\Omega_t))}\leqslant \varepsilon.$$
故自由项$f$在$L^2(K)$意义下作微小改变时, 对应的柯西问题的解$u$在$L^2(K)$意义下改变也是微小的.

\section{2.1/2}
\begin{problem}
试直接推导扩散过程所满足的微分方程.
\end{problem}
设$N(x,y,t)$表示扩散物质的浓度, $dm$表示在无穷小时段$dt$内沿法线方向$\mathbf{n}$经过一个无穷小面积$dS$的扩散物质的质量, $D(x,y,t)$为扩散系数, 其总取正值, 则
$$dm=-D(x,y,t)\frac{\partial N(x,y,t)}{\partial \mathbf{n}}dSdt.$$
故从$t_1$到$t_2$进入扩散面为$\Gamma$的区域$\Omega$的质量为
$$\int_{t_1}^{t_2}\iint\limits_{\Gamma}-dm=\int_{t_1}^{t_2}\iint\limits_{\Gamma}D\frac{\partial N}{\partial t}dSdt=\int_{t_1}^{t_2}\iiint\limits_{\Omega}\left[\frac{\partial}{\partial x}\left(D\frac{\partial N}{\partial x}\right)+\frac{\partial}{\partial y}\left(D\frac{\partial N}{\partial y}\right)+\frac{\partial}{\partial z}\left(D\frac{\partial N}{\partial z}\right)\right]dxdydzdt.$$
区域$\Omega$内, 从$t_1$到$t_2$物质的增加量也可表示为
$$\iiint\limits_{\Omega}\left[N(x,y,z,t_2)-N(x,y,z,t_1)\right]dxdydz=\iiint\limits_{\Omega}\int_{t_1}^{t_2}\frac{\partial N}{\partial t}dtdxdydz=\int_{t_1}^{t_2}\iiint\limits_{\Omega}\frac{\partial N}{\partial t}dxdydzdt.$$
由于这两个质量相等, 且$t_1$, $t_2$, $\Omega$的取值是任意的, 可得
$$\frac{\partial N}{\partial t}=\frac{\partial}{\partial x}\left(D\frac{\partial N}{\partial x}\right)+\frac{\partial}{\partial y}\left(D\frac{\partial N}{\partial y}\right)+\frac{\partial}{\partial z}\left(D\frac{\partial N}{\partial z}\right).$$

\section{2.2/1}
\begin{problem}
用分离变量法求下列定解问题的解:
$$\left\{\begin{aligned}
     & \frac{\partial u}{\partial t}=a^2\frac{\partial^2u}{\partial t^2}\quad (t>0,0<x<\pi), \\
     & u(0,t)=\frac{\partial u}{\partial x}(\pi,t)=0\quad (t>0),                             \\
     & u(x,0)=f(x)\quad (0<x<\pi).
  \end{aligned}\right.$$
\end{problem}

方程的特征值和对应的特征函数为
$$X''(x)+\lambda X(x)=0,$$
$$X(x)=C_1\cos\sqrt{\lambda}x+C_2\sin\sqrt{\lambda}x,$$

代入$X(0)=0$, $X'(\pi)=0$得
$$C_1=0,\quad C_2\sqrt{\lambda}\cos\sqrt{\lambda}\pi=0,$$
$$\lambda=\lambda_k=\frac{(2k-1)^2}{4},\quad X_k(x)=C_k\sin\frac{(2k-1)}{2}x,\quad k=1,2,\cdots.$$

\section{2.2/2}


\end{document}
