\documentclass[11pt,a4paper]{article}
\usepackage{../ma319}
\semester{Fall}
\year{2019}
\subtitlenumber{9}
\author{刘逸灏 (515370910207)}

\begin{document}

\maketitle

\section{2.5/1}
\begin{problem}
证明下列热传导方程初边值问题
$$\left\{\begin{aligned}
     & u_t-a^2u_{xx}=0,     \\
     & u|_{x=0}=u|_{x=l}=0, \\
     & u|_{t=0}=\varphi(x)
  \end{aligned}\right.$$
的解当$t\to+\infty$时指数地衰减为零, 其中$\varphi\in C^2$, 且$\varphi(0)=\varphi(l)=0$.
\end{problem}

方程的解是
$$u(x,t)=\sum_{k=1}^\infty A_ke^{-\frac{k^2\pi^2a^2}{l^2}t}\sin\frac{k\pi}{l}x,\quad A_k=\frac{2}{l}\int_0^l\varphi(x)\sin\frac{k\pi}{l}xdx.$$
其中
$$\left|\sin\frac{k\pi}{l}x\right|\leqslant1,$$
$$|A_k|\leqslant\frac{2}{l}\cdot l\cdot\max_{0\leqslant x\leqslant l}\left|\varphi(x)\sin\frac{k\pi}{l}x\right|\leqslant 2\max_{0\leqslant x\leqslant l}|\varphi(x)|=C$$
由$\varphi\in C^2$且$\varphi(0)=\varphi(l)=0$可知$C$是一个确定的正常数.
$$|u(x,t)|\leqslant\sum_{k=1}^\infty Ce^{-\frac{k^2\pi^2a^2}{l^2}t}=C\left(1+\sum_{k=2}^\infty e^{-\frac{(k^2-1)\pi^2a^2}{l^2}t}\right)e^{-\frac{\pi^2a^2}{l^2}t}.$$
当$t\to+\infty$时
$$\sum_{k=2}^\infty e^{-\frac{(k^2-1)\pi^2a^2}{l^2}t}\leqslant \sum_{k=1}^\infty e^{-\frac{k^2\pi^2a^2}{l^2}t}\leqslant\int_0^\infty e^{-\frac{k^2\pi^2a^2}{l^2}t}dk=\frac{\sqrt{\pi}}{2}\sqrt{\frac{l^2}{\pi^2a^2t}}\leqslant C_1.$$
设$C_2=C(1+C_1)$即可得
$$|u(x,t)|\leqslant C_2e^{-\frac{\pi^2a^2}{l^2}t}.$$

\section{2.5/2}
\begin{problem}
证明: 当$\varphi(x,y)$为$\mathbf{R}^2$上的有界连续函数, 且$\varphi\in L^1(\mathbf{R}^2)$时, 二维热传导方程柯西问题的解, 当$t\to+\infty$时, 以$t^{-1}$衰减率趋于零.
\end{problem}

方程的解是
$$u(x,y,t)=\frac{1}{4\pi a^2t}\int_{-\infty}^\infty\int_{-\infty}^\infty\varphi(\xi,\eta)e^{-\frac{(x-\xi)^2+(y-\eta)^2}{4a^2t}}d\xi d\eta.$$
由$\varphi(x,y)$为$\mathbf{R}^2$上的有界连续函数, 且$\varphi\in L^1(\mathbf{R}^2)$可知
$$\int_{-\infty}^\infty\int_{-\infty}^\infty|\varphi(\xi,\eta)|d\xi d\eta\leqslant C.$$
当$t\to+\infty$时
$$\left|e^{-\frac{(x-\xi)^2+(y-\eta)^2}{4a^2t}}\right|\leqslant1,$$
$$|u(x,y,t)|\leqslant \frac{1}{4\pi a^2t}\int_{-\infty}^\infty\int_{-\infty}^\infty\left|\varphi(\xi,\eta)e^{-\frac{(x-\xi)^2+(y-\eta)^2}{4a^2t}}\right|d\xi d\eta\leqslant\frac{C}{4\pi a^2t}.$$
设$C_1=\dfrac{C}{4\pi a^2}$即可得
$$|u(x,y,t)|\leqslant C_1t^{-1}.$$

\section{2.5/3}
\begin{problem}
证明: 当$\varphi(x,y,z)$为$\mathbf{R}^3$上的有界连续函数, 且$\varphi\in L^1(\mathbf{R}^3)$时, 三维热传导方程柯西问题的解, 当$t\to+\infty$时, 以$t^{-\frac{3}{2}}$衰减率趋于零.
\end{problem}

方程的解是
$$u(x,y,z,t)=\frac{1}{8a^3\sqrt{\pi t}^3}\int_{-\infty}^\infty\int_{-\infty}^\infty\int_{-\infty}^\infty\varphi(\xi,\eta,\zeta)e^{-\frac{(x-\xi)^2+(y-\eta)^2+(z-\zeta)^2}{4a^2t}}d\xi d\eta d\zeta.$$
由$\varphi(x,y,z)$为$\mathbf{R}^2$上的有界连续函数, 且$\varphi\in L^1(\mathbf{R}^3)$可知
$$\int_{-\infty}^\infty\int_{-\infty}^\infty\int_{-\infty}^\infty|\varphi(\xi,\eta,\zeta)|d\xi d\eta d\zeta\leqslant C.$$
当$t\to+\infty$时
$$\left|e^{-\frac{(x-\xi)^2+(y-\eta)^2+(z-\zeta)^2}{4a^2t}}\right|\leqslant1,$$
$$|u(x,y,t)|\leqslant \frac{1}{8a^3\sqrt{\pi t}^3}\int_{-\infty}^\infty\int_{-\infty}^\infty\int_{-\infty}^\infty\left|\varphi(\xi,\eta,\zeta)e^{-\frac{(x-\xi)^2+(y-\eta)^2+(z-\zeta)^2}{4a^2t}}\right|d\xi d\eta d\zeta\leqslant\frac{C}{8a^3\sqrt{\pi t}^3}.$$
设$C_1=\dfrac{C}{8a^3\sqrt{\pi}^3}$即可得
$$|u(x,y,t)|\leqslant C_1t^{-\frac{3}{2}}.$$

\section{3.1/1}
\begin{problem}
设$u(x_1,\cdots,x_n)=f(r)$(其中$r=\sqrt{x_1^2+\cdots+x_n^2}$)是$n$维调和函数(即满足方程$\dfrac{\partial^2u}{\partial x_1^2}+\cdots+\dfrac{\partial^2u}{\partial x_n^2}=0$), 试证明
$$f(r)=c_1+\frac{c_2}{r^{n-2}}\quad(n\neq 2),$$
$$f(r)=c_1+c_2\ln\frac{1}{r}\quad(n=2),$$
其中$c_1$, $c_2$为任意常数.
\end{problem}

$$\frac{\partial r}{\partial x_i}=\frac{\partial}{\partial x_i}\sqrt{x_1^2+\cdots+x_n^2}=\frac{1}{2\sqrt{x_1^2+\cdots+x_n^2}}\cdot 2x_i=\frac{x_i}{r},$$
$$\frac{\partial u}{\partial x_i}=\frac{\partial f(r)}{\partial r}\cdot\frac{\partial r}{\partial x_i}=f'(r)\frac{x_i}{r},$$
$$\frac{\partial^2u}{\partial x_i^2}=f'(r)\frac{1}{r}\frac{\partial x_i}{\partial x_i}+x_i\frac{\partial}{\partial r}\left[f'(r)\frac{1}{r}\right]\frac{\partial r}{\partial x_i}=f'(r)\frac{1}{r}+\left[f''(r)\frac{x_i}{r}-f'(r)\frac{x_i}{r^2}\right]\frac{x_i}{r}=f''(r)\frac{x_i^2}{r^2}+\left(\frac{1}{r}-\frac{x_i^2}{r^3}\right)f'(r),$$
$$\sum_{i=1}^n\frac{\partial^2u}{\partial x_i^2}=f''(r)+\frac{n-1}{r}f'(r)=0,$$
$$\frac{f''(r)}{f'(r)}=\frac{\frac{df'(r)}{dr}}{f'(r)}=-\frac{n-1}{r},$$
$$\frac{df'(r)}{f'(r)}=-\frac{n-1}{r}dr.$$
两边积分得
$$\int\frac{1}{f'(r)}df'(r)=-\int\frac{n-1}{r}dr,$$
$$\ln f'(r)=-(n-1)\ln r+c_0,$$
$$f'(r)=c_0e^{-(n-1)\ln r}=\frac{c_0}{r^{n-1}}.$$
再次积分得
$$f(r)=\left\{\begin{aligned}
     & c_1+\frac{c_2}{r^{n-2}} & \quad (n\neq2), \\
     & c_1+c_2\ln\frac{1}{r}   & \quad (n=2).    \\
  \end{aligned}\right.$$

\section{3.1/3}
\begin{problem}
证明: 拉普拉斯算子在柱坐标$(r,\theta,z)$下可以写为
$$\Delta u=\frac{1}{r}\frac{\partial}{\partial r}\left(r\frac{\partial u}{\partial r}\right)+\frac{1}{r^2}\frac{\partial^2u}{\partial\theta^2}+\frac{\partial^2u}{\partial z^2}.$$
\end{problem}

$$r=\sqrt{x^2+y^2},\quad \theta=\arctan\frac{y}{x},$$
$$\frac{\partial r}{\partial x}=\frac{x}{r}=\cos\theta,\quad\frac{\partial r}{\partial y}=\frac{y}{r}=\sin\theta,$$
$$\frac{\partial\theta}{\partial x}=-\frac{y}{x^2+y^2}=-\frac{y}{r^2}=-\frac{\sin\theta}{r},\quad\frac{\partial\theta}{\partial y}=\frac{x}{x^2+y^2}=\frac{x}{r^2}=\frac{\cos\theta}{r}.$$
则
$$\frac{\partial u}{\partial x}=\frac{\partial u}{\partial r}\frac{\partial r}{\partial x}+\frac{\partial u}{\partial\theta}\frac{\partial\theta}{\partial x}=\frac{\partial u}{\partial r}\cos\theta-\frac{\partial u}{\partial \theta}\frac{\sin\theta}{r},$$
$$\frac{\partial u}{\partial y}=\frac{\partial u}{\partial r}\frac{\partial r}{\partial y}+\frac{\partial u}{\partial\theta}\frac{\partial\theta}{\partial y}=\frac{\partial u}{\partial r}\sin\theta+\frac{\partial u}{\partial \theta}\frac{\cos\theta}{r}.$$
\begin{align*}
  \frac{\partial^2u}{\partial x^2}
   & =\frac{\partial}{\partial r}\left(\frac{\partial u}{\partial x}\right)\frac{\partial r}{\partial x}-\frac{\partial}{\partial\theta}\left(\frac{\partial u}{\partial x}\right)\frac{\partial\theta}{\partial x}                                                                                                 \\
   & =\frac{\partial^2u}{\partial r^2}\cos^2\theta-\frac{\partial^2u}{\partial r\partial\theta}\frac{\sin\theta\cos\theta}{r}+\frac{\partial u}{\partial\theta}\frac{\sin\theta\cos\theta}{r^2}
  -\frac{\partial^2u}{\partial r\partial\theta}\frac{\sin\theta\cos\theta}{r}+\frac{\partial u}{\partial r}\frac{\sin^2\theta}{r}+\frac{\partial^2u}{\partial\theta^2}\frac{\sin^2\theta}{r^2}+\frac{\partial u}{\partial\theta}\frac{\sin\theta\cos\theta}{r^2}                                                    \\
   & =\frac{\partial^2u}{\partial r^2}\cos^2\theta-\frac{\partial^2u}{\partial r\partial\theta}\frac{2\sin\theta\cos\theta}{r}+\frac{\partial^2u}{\partial\theta^2}\frac{\sin^2\theta}{r^2}+\frac{\partial u}{\partial r}\frac{\sin^2\theta}{r}+\frac{\partial u}{\partial\theta}\frac{2\sin\theta\cos\theta}{r^2},
\end{align*}
\begin{align*}
  \frac{\partial^2u}{\partial y^2}
   & =\frac{\partial}{\partial r}\left(\frac{\partial u}{\partial y}\right)\frac{\partial r}{\partial y}-\frac{\partial}{\partial\theta}\left(\frac{\partial u}{\partial y}\right)\frac{\partial\theta}{\partial y}                                                                                                 \\
   & =\frac{\partial^2u}{\partial r^2}\sin^2\theta+\frac{\partial^2u}{\partial r\partial\theta}\frac{\sin\theta\cos\theta}{r}-\frac{\partial u}{\partial\theta}\frac{\sin\theta\cos\theta}{r^2}
  +\frac{\partial^2u}{\partial r\partial\theta}\frac{\sin\theta\cos\theta}{r}+\frac{\partial u}{\partial r}\frac{\cos^2\theta}{r}+\frac{\partial^2u}{\partial\theta^2}\frac{\cos^2\theta}{r^2}-\frac{\partial u}{\partial\theta}\frac{\sin\theta\cos\theta}{r^2}                                                    \\
   & =\frac{\partial^2u}{\partial r^2}\sin^2\theta+\frac{\partial^2u}{\partial r\partial\theta}\frac{2\sin\theta\cos\theta}{r}+\frac{\partial^2u}{\partial\theta^2}\frac{\cos^2\theta}{r^2}+\frac{\partial u}{\partial r}\frac{\cos^2\theta}{r}-\frac{\partial u}{\partial\theta}\frac{2\sin\theta\cos\theta}{r^2}.
\end{align*}
故
$$\Delta u=\frac{\partial^2u}{\partial x^2}+\frac{\partial^2u}{\partial y^2}+\frac{\partial^2u}{\partial z^2}
  =\frac{\partial^2u}{\partial r^2}+\frac{\partial^2u}{\partial\theta^2}\frac{1}{r^2}+\frac{\partial u}{\partial r}\frac{1}{r}+\frac{\partial^2u}{\partial z^2}
  =\frac{1}{r}\frac{\partial}{\partial r}\left(r\frac{\partial u}{\partial r}\right)+\frac{1}{r^2}\frac{\partial^2u}{\partial\theta^2}+\frac{\partial^2u}{\partial z^2}.$$

\section{3.1/5}
证明用极坐标表示的下列函数都满足调和方程:
\begin{problem}
\begin{enumerate}
  \item $\ln r$和$\theta$;
  \item $r^n\cos n\theta$和$r^n\sin n\theta$;
  \item $r\ln r\cos\theta-r\theta\sin\theta$和$r\ln r\sin\theta+r\theta\cos\theta$.
\end{enumerate}
\end{problem}

由上题易知拉普拉斯算子在极坐标系$(r,\theta)$下可以写为
$$\Delta u=\frac{\partial^2u}{\partial x^2}+\frac{\partial^2u}{\partial y^2}=\frac{1}{r}\frac{\partial}{\partial r}\left(r\frac{\partial u}{\partial r}\right)+\frac{1}{r^2}\frac{\partial^2u}{\partial\theta^2}.$$

\subsection*{(1)}
$$\left(\frac{\partial^2}{\partial x^2}+\frac{\partial^2}{\partial y^2}\right)\ln r=\frac{1}{r}\frac{\partial}{\partial r}(1)=0,$$
$$\left(\frac{\partial^2}{\partial x^2}+\frac{\partial^2}{\partial y^2}\right)\theta=\frac{1}{r^2}\frac{\partial}{\partial\theta}(1)=0.$$

\subsection*{(2)}
$$\left(\frac{\partial^2}{\partial x^2}+\frac{\partial^2}{\partial y^2}\right)r^n\cos n\theta=\frac{1}{r}\frac{\partial}{\partial r}nr^n\cos n\theta-\frac{1}{r^2}\frac{\partial}{\partial\theta}nr^n\sin n\theta=n^2r^{n-2}\cos n\theta-n^2r^{n-2}\cos n\theta=0,$$
$$\left(\frac{\partial^2}{\partial x^2}+\frac{\partial^2}{\partial y^2}\right)r^n\sin n\theta=\frac{1}{r}\frac{\partial}{\partial r}nr^n\sin n\theta+\frac{1}{r^2}\frac{\partial}{\partial\theta}nr^n\cos n\theta=n^2r^{n-2}\sin n\theta-n^2r^{n-2}\sin n\theta=0.$$

\subsection*{(3)}
\begin{align*}
    & \left(\frac{\partial^2}{\partial x^2}+\frac{\partial^2}{\partial y^2}\right)(r\ln r\cos\theta-r\theta\sin\theta)                                                                   \\
  = & \frac{1}{r}\frac{\partial}{\partial r}r(\cos\theta+\cos\theta\ln r-\theta\sin\theta)+\frac{1}{r^2}\frac{\partial}{\partial\theta}(-r\theta\cos\theta-r\sin\theta-r\ln r\sin\theta) \\
  = & \frac{2\cos\theta+\cos\theta\ln r-\theta\sin\theta}{2}-\frac{2\cos\theta+\cos\theta\ln r-\theta\sin\theta}{2}                                                                      \\
  = & 0,
\end{align*}
\begin{align*}
    & \left(\frac{\partial^2}{\partial x^2}+\frac{\partial^2}{\partial y^2}\right)(r\ln r\cos\theta-r\theta\sin\theta)                                                                   \\
  = & \frac{1}{r}\frac{\partial}{\partial r}r(\sin\theta+\sin\theta\ln r+\theta\cos\theta)+\frac{1}{r^2}\frac{\partial}{\partial\theta}(-r\theta\sin\theta+r\cos\theta+r\ln r\cos\theta) \\
  = & \frac{2\sin\theta+\sin\theta\ln r+\theta\cos\theta}{2}-\frac{2\sin\theta+\sin\theta\ln r+\theta\cos\theta}{2}                                                                      \\
  = & 0.
\end{align*}

\end{document}
