\documentclass[11pt,a4paper]{article}
\usepackage{../ma319}
\semester{Fall}
\year{2019}
\subtitlenumber{3}
\author{刘逸灏 (515370910207)}

\begin{document}
\maketitle

\section{1.3/1}

\begin{problem}
用分离变量法求下列问题的解:
\begin{enumerate}
  \item $
          \left\{\begin{aligned}
             & \frac{\partial^2u}{\partial t^2}=a^2\frac{\partial^2u}{\partial x^2},                                      \\
             & u|_{t=0}=\sin\frac{3\pi x}{l},\quad \left.\frac{\partial u}{\partial t}\right|_{t=0}=x(l-x) \quad (0<x<l), \\
             & u(0,t)=u(l,t)=0;
          \end{aligned}\right.
        $
  \item $
          \left\{\begin{aligned}
             & \frac{\partial^2u}{\partial t^2}-a^2\frac{\partial^2u}{\partial x^2}=0 \\
             & u(0,t)=0,\quad \frac{\partial u}{\partial x}(l,t)=0,                   \\
             & u(x,0)=\frac{h}{l}x,                                                   \\
             & \frac{\partial u}{\partial x}(x,0)=0.                                  \\
          \end{aligned}\right..
        $
\end{enumerate}
\end{problem}

\subsection*{(1)}

方程的特征值和对应的特征函数为
$$\lambda=\lambda_k=\frac{k^2\pi^2}{l^2},\quad X_k(x)=C_k\sin\sqrt{\lambda}x=C_k\sin\frac{k\pi}{l}x,\quad k=1,2,\cdots.$$

方程的通解为
\begin{align*}
  u(x,t)
   & =\sum_{k=1}^\infty\left(A_k\cos\sqrt{\lambda}at+
  B_k\sin\sqrt{\lambda}at\right)\sin\sqrt{\lambda}x    \\
   & =\sum_{k=1}^\infty\left(A_k\cos\frac{k\pi a}{l}t+
  B_k\sin\frac{k\pi a}{l}t\right)\sin\frac{k\pi}{l}x.
\end{align*}

代入初值条件得
$$\varphi(x)=\sin\frac{3\pi x}{l}=\sum_{k=1}^\infty A_k\sin\sqrt{\lambda}x=\sum_{k=1}^\infty A_k\sin\frac{k\pi }{l}x,$$
$$\psi(x)=x(l-x)=\sum_{k=1}^\infty B_k\sqrt{\lambda}a\sin\sqrt{\lambda}x=\sum_{k=1}^\infty \frac{B_kk\pi a}{l}\sin\frac{k\pi}{l}x.$$

解得
$$A_k=\left\{\begin{aligned}1, &\quad k = 3 \\ 0, &\quad k\neq 3 \end{aligned}\right.,$$
$$B_k=\frac{2}{l\sqrt{\lambda}a}\int_0^l\psi(x)\sin\sqrt{\lambda}xdx=
  \frac{2}{k\pi a}\int_0^l x(l-x)\sin\frac{k\pi}{l}xdx=-\frac{2l^3(-2+2\cos k\pi+k\pi\sin k\pi)}{k^4\pi^4a}.$$

化简得
$$B_k=\left\{\begin{aligned}&\frac{8l^3}{k^4\pi^4a}, & k = 2n-1 \\ &0, & k=2n \end{aligned}\right.,\quad n=1,2,\cdots.$$

故
$$u(x,t)=\cos\frac{3\pi a}{l}t\sin\frac{3\pi}{l}x+\sum_{n=1}^\infty\frac{8l^3}{(2n-1)^4\pi^4a}\sin\frac{(2n-1)\pi a}{l}t\sin\frac{(2n-1)\pi}{l}x.$$

\subsection*{(ii)}

方程的特征值和对应的特征函数为
$$X''(x)+\lambda X(x)=0,$$
$$X(x)=C_1\cos\sqrt{\lambda}x+C_2\sin\sqrt{\lambda}x,$$

代入$X(0)=0$, $X'(l)=0$得
$$C_1=0,\quad C_2\sqrt{\lambda}\cos\sqrt{\lambda}l=0,$$
$$\lambda=\lambda_k=\frac{(2k-1)^2\pi^2}{4l^2},\quad X_k(x)=C_k\sin\frac{(2k-1)\pi}{2l}x,\quad k=1,2,\cdots.$$

方程的通解为
\begin{align*}
  u(x,t)
   & =\sum_{k=1}^\infty\left(A_k\cos\sqrt{\lambda}at+
  B_k\sin\sqrt{\lambda}at\right)\sin\sqrt{\lambda}x          \\
   & =\sum_{k=1}^\infty\left(A_k\cos\frac{(2k-1)\pi a}{2l}t+
  B_k\sin\frac{(2k-1)\pi a}{2l}t\right)\sin\frac{(2k-1)\pi}{2l}x.
\end{align*}

代入初值条件得
$$\varphi(x)=\frac{h}{l}x=\sum_{k=1}^\infty A_k\sin\sqrt{\lambda}x=\sum_{k=1}^\infty A_k\sin\frac{(2k-1)\pi }{2l}x,$$
$$\psi(x)=0=\sum_{k=1}^\infty B_k\sqrt{\lambda}a\sin\sqrt{\lambda}x=\sum_{k=1}^\infty \frac{B_k(2k-1)\pi a}{2l}\sin\frac{(2k-1)\pi}{2l}x.$$

解得
$$A_k=\frac{2}{l}\int_0^l\varphi(x)\sin\sqrt{\lambda}xdx=\frac{2}{l}\int_0^l\frac{h}{l}x\sin\frac{(2k-1)\pi}{2l}xdx=-\frac{4h[2\cos k\pi+(2k-1)\pi\sin k\pi]}{(2k-1)^2\pi^2},$$
$$B_k=0.$$

化简得
$$A_k=(-1)^{k+1}\frac{8h}{(2k-1)^2\pi^2}.$$

故
$$u(x,t)=\sum_{k=1}^\infty (-1)^{k+1}\frac{8h}{(2k-1)^2\pi^2}\cos\frac{(2k-1)\pi a}{2l}t\sin\frac{(2k-1)\pi}{2l}x.$$

\section{1.3/2}
\begin{problem}
设弹簧一端固定, 一端在外力作用下作周期运动, 此时定解问题归结为
$$
  \left\{\begin{aligned}
     & \frac{\partial^2u}{\partial t^2}=a^2\frac{\partial^2u}{\partial x^2}, \\
     & u(0,t)=0,\quad u(l,t)=A\sin\omega t                                   \\
     & u(x,0)=\frac{\partial u}{\partial t}(x,0)=0.
  \end{aligned}\right.
$$
求解此问题.
\end{problem}



设方程的一个特解为
$$U(x,t)=X(x)\sin\omega t,$$
$$U_{tt}=-\omega^2X(x)\sin\omega t,$$
$$U_{xx}=X''(x)\sin\omega t.$$

代入方程和边界条件得
$$X''+\frac{\omega^2}{a^2}X=0,\quad X(0)=0,\quad X(l)=A.$$
$$X(x)=C_1\cos\frac{\omega}{a}x+C_2\sin\frac{\omega}{a}x,$$
$$C_1=0,\quad C_2\sin\frac{\omega}{a}l=A\Longrightarrow
  C_2=\frac{A}{\sin\frac{\omega}{a}l},$$
$$U(x,t)=X(x)\sin\omega t=\frac{A}{\sin\frac{\omega}{a}l}\sin\frac{\omega}{a}x\sin\omega t.$$

此时
$$U(0,t)=0,\quad U(l,t)=A\sin\omega t,\quad U(x,0)=0,\quad \frac{\partial U}{\partial t}(x,0)=\frac{A\omega}{\sin\frac{\omega}{a}l}\sin\frac{\omega}{a}x.$$

方程的通解为
$$u(x,t)=U(x,t)+V(x,t).$$

且$V(x,t)$满足如下初边值问题
$$
  \left\{\begin{aligned}
     & \frac{\partial^2V}{\partial t^2}=a^2\frac{\partial^2V}{\partial x^2}, \\
     & V(0,t)=0,\quad V(l,t)=0                                               \\
     & V(x,0)=0,\quad \frac{\partial V}{\partial t}(x,0)
    =-\frac{A\omega}{\sin\frac{\omega}{a}l}\sin\frac{\omega}{a}x.
  \end{aligned}\right.
$$

$V(x,t)$的特征值和对应的特征函数为
$$\lambda=\lambda_k=\frac{k^2\pi^2}{l^2},\quad X_k(x)=C_k\sin\sqrt{\lambda}x=C_k\sin\frac{k\pi}{l}x,\quad k=1,2,\cdots.$$

$V(x,t)$的通解为
\begin{align*}
  V(x,t)
   & =\sum_{k=1}^\infty\left(A_k\cos\sqrt{\lambda}at+
  B_k\sin\sqrt{\lambda}at\right)\sin\sqrt{\lambda}x    \\
   & =\sum_{k=1}^\infty\left(A_k\cos\frac{k\pi a}{l}t+
  B_k\sin\frac{k\pi a}{l}t\right)\sin\frac{k\pi}{l}x.
\end{align*}

代入初值条件得
$$\varphi(x)=0=\sum_{k=1}^\infty A_k\sin\sqrt{\lambda}x=\sum_{k=1}^\infty A_k\sin\frac{k\pi }{l}x,$$
$$\psi(x)=-\frac{A\omega}{\sin\frac{\omega}{a}l}\sin\frac{\omega}{a}x=\sum_{k=1}^\infty B_k\sqrt{\lambda}a\sin\sqrt{\lambda}x=\sum_{k=1}^\infty \frac{B_kk\pi a}{l}\sin\frac{k\pi}{l}x.$$

解得
$$A_k=0,$$
\begin{align*}
  B_k & =\frac{2}{l\sqrt{\lambda}a}\int_0^l\psi(x)\sin\sqrt{\lambda}xdx=
  \frac{2}{k\pi a}\int_0^l -\frac{A\omega}{\sin\frac{\omega}{a}l}\sin\frac{\omega}{a}x\sin\frac{k\pi}{l}xdx                                                                         \\
      & =-\frac{2A\omega}{k\pi\sin\frac{\omega}{a}l}\cdot\frac{l\left(\omega l\cos\frac{\omega}{a}l\sin k\pi-k\pi a\sin\frac{\omega}{a}l\cos k\pi\right)}{k^2\pi^2a^2-\omega^2l^2}.
\end{align*}

化简得
$$B_k=(-1)^k\frac{2A\omega al}{\omega^2l^2-k^2\pi^2a^2}.$$

故
$$V(x,t)=\sum_{k=1}^\infty(-1)^k\frac{2A\omega al}{\omega^2l^2-k^2\pi^2a^2}\sin\frac{k\pi a}{l}t\sin\frac{k\pi}{l}x,$$
$$u(x,t)=U(x,t)+V(x,t)=\frac{A}{\sin\frac{\omega}{a}l}\sin\frac{\omega}{a}x\sin\omega t+\sum_{k=1}^\infty(-1)^k\frac{2A\omega al}{\omega^2l^2-k^2\pi^2a^2}\sin\frac{k\pi a}{l}t\sin\frac{k\pi}{l}x.$$

\section{1.3/3}
\begin{problem}
求弦震动方程
$$u_{tt}-a^2u_{xx}=0, \quad 0<x<l,\quad t>0$$
满足以下定解条件的解
\begin{enumerate}
  \item $u|_{x=0}=u_x|_{x=l}=0,$\\
        $u|_{t=0}=\sin\frac{3}{2l}\pi x,\quad u_t|_{t=0}=\sin\frac{5}{2l}\pi x;$
  \item $u_x|_{x=0}=u_x|_{x=l}=0,\\$
        $u|_{t=0}=x,\quad u_t|_{t=0}=0.$
\end{enumerate}
\end{problem}
\subsection*{(1)}

$$
  \left\{\begin{aligned}
     & u_{tt}-a^2u_{xx}=0, \quad 0<x<l,\quad t>0,                             \\
     & u|_{x=0}=u_x|_{x=l}=0,                                                 \\
     & u|_{t=0}=\sin\frac{3}{2l}\pi x,\quad u_t|_{t=0}=\sin\frac{5}{2l}\pi x.
  \end{aligned}\right.
$$

方程的特征值和对应的特征函数为
$$X''(x)+\lambda X(x)=0,$$
$$X(x)=C_1\cos\sqrt{\lambda}x+C_2\sin\sqrt{\lambda}x,$$

代入$X(0)=0$, $X'(l)=0$得
$$C_1=0,\quad C_2\sqrt{\lambda}\cos\sqrt{\lambda}l=0,$$
$$\lambda=\lambda_k=\frac{(2k-1)^2\pi^2}{4l^2},\quad X_k(x)=C_k\sin\frac{(2k-1)\pi}{2l}x,\quad k=1,2,\cdots.$$

方程的通解为
\begin{align*}
  u(x,t)
   & =\sum_{k=1}^\infty\left(A_k\cos\sqrt{\lambda}at+
  B_k\sin\sqrt{\lambda}at\right)\sin\sqrt{\lambda}x          \\
   & =\sum_{k=1}^\infty\left(A_k\cos\frac{(2k-1)\pi a}{2l}t+
  B_k\sin\frac{(2k-1)\pi a}{2l}t\right)\sin\frac{(2k-1)\pi}{2l}x.
\end{align*}

代入初值条件得
$$\varphi(x)=\sin\frac{3}{2l}\pi x=\sum_{k=1}^\infty A_k\sin\sqrt{\lambda}x=\sum_{k=1}^\infty A_k\sin\frac{(2k-1)\pi }{2l}x,$$
$$\psi(x)=\sin\frac{5}{2l}\pi x=\sum_{k=1}^\infty B_k\sqrt{\lambda}a\sin\sqrt{\lambda}x=\sum_{k=1}^\infty \frac{B_k(2k-1)\pi a}{2l}\sin\frac{(2k-1)\pi}{2l}x.$$

解得
$$A_k=\left\{\begin{aligned}1, &\quad k = 2 \\ 0, &\quad k\neq 2 \end{aligned}\right.,\quad B_k=\left\{\begin{aligned}&\frac{2l}{5\pi a}, &\quad k = 3 \\ &0, &\quad k\neq 3 \end{aligned}\right..$$

故
$$u(x,t)=\cos\frac{3\pi a}{2l}t\sin\frac{3\pi}{2l}x+
  \frac{2l}{5\pi a}\sin\frac{5\pi a}{2l}t\sin\frac{5\pi}{2l}x.$$

\subsection*{(2)}

$$
  \left\{\begin{aligned}
     & u_{tt}-a^2u_{xx}=0, \quad 0<x<l,\quad t>0, \\
     & u_x|_{x=0}=u_x|_{x=l}=0,                   \\
     & u|_{t=0}=x,\quad u_t|_{t=0}=0.
  \end{aligned}\right.
$$

方程的特征值和对应的特征函数为
$$X''(x)+\lambda X(x)=0,$$
$$X(x)=C_1\cos\sqrt{\lambda}x+C_2\sin\sqrt{\lambda}x,$$

代入$X'(0)=0$, $X'(l)=0$得
$$C_2=0,\quad -C_1\sqrt{\lambda}\sin\sqrt{\lambda}l=0,$$
$$\lambda=\lambda_k=\frac{k^2\pi^2}{l^2},\quad X_k(x)=C_k\cos\frac{k\pi}{l}x,\quad k=0,1,2,\cdots.$$

方程的通解为
\begin{align*}
  u(x,t)
   & =\frac{1}{2}A_0+\frac{1}{2}B_0t+
  \sum_{k=1}^\infty\left(A_k\cos\sqrt{\lambda}at+
  B_k\sin\sqrt{\lambda}at\right)\cos\sqrt{\lambda}x \\
   & =\frac{1}{2}A_0+\frac{1}{2}B_0t+
  \sum_{k=1}^\infty\left(A_k\cos\frac{k\pi a}{l}t+
  B_k\sin\frac{k\pi a}{l}t\right)\cos\frac{k\pi}{l}x.
\end{align*}

代入初值条件得
$$\varphi(x)=x=\frac{1}{2}A_0+\sum_{k=1}^\infty A_k\cos\sqrt{\lambda}x=\frac{1}{2}A_0+\sum_{k=1}^\infty A_k\cos\frac{k\pi }{l}x,$$
$$\psi(x)=0=\frac{1}{2}B_0+\sum_{k=1}^\infty B_k\sqrt{\lambda}a\cos\sqrt{\lambda}x=\frac{1}{2}B_0+\sum_{k=1}^\infty \frac{B_kk\pi a}{l}\cos\frac{k\pi}{l}x.$$

解得
$$A_k=\frac{2}{l}\int_0^l\varphi(x)\cos\sqrt{\lambda}xdx=\frac{2}{l}\int_0^lx\cos\frac{k\pi}{l}xdx=\left\{\begin{aligned}&l,&\quad k=0\\&\frac{2l(-1+\cos k\pi+k\pi\sin k\pi)}{k^2\pi^2},&\quad k\neq 0\end{aligned}\right.,$$
$$B_k=0.$$

化简得
$$A_k=\left\{\begin{aligned}&l,&\quad k=0\\&-\frac{4l}{k^2\pi^2}, &\quad k = 2n-1 \\ &0, &\quad k=2n \end{aligned}\right.,\quad n=1,2,\cdots.$$

故
$$u(x,t)=\frac{l}{2}+\sum_{n=1}^\infty-\frac{4l}{(2n-1)^2\pi^2}\cos\frac{(2n-1)\pi a}{l}t\cos\frac{(2n-1)\pi}{l}x.$$

\section{1.3/4}
\begin{problem}
用分离变量法求解初边值问题:
$$
  \left\{\begin{aligned}
     & u_{tt}-a^2u_{xx}=g, \quad 0<x<l,\quad t>0,        \\
     & u|_{x=0}=u_x|_{x=l}=0,                            \\
     & u|_{t=0}=0,\quad u_t|_{t=0}=\sin\frac{\pi x}{2l}.
  \end{aligned}\right.
$$
其中$g$是常数.
\end{problem}


利用叠加原理, 设$$u(x,t)=U(x,t)+V(x,t).$$

先求
$$
  \left\{\begin{aligned}
     & U_{tt}-a^2U_{xx}=0, \quad 0<x<l,\quad t>0,        \\
     & U|_{x=0}=U_x|_{x=l}=0,                            \\
     & U|_{t=0}=0,\quad U_t|_{t=0}=\sin\frac{\pi x}{2l}.
  \end{aligned}\right.
$$

方程的特征值和对应的特征函数为
$$X''(x)+\lambda X(x)=0,$$
$$X(x)=C_1\cos\sqrt{\lambda}x+C_2\sin\sqrt{\lambda}x,$$

代入$X(0)=0$, $X'(l)=0$得
$$C_1=0,\quad C_2\sqrt{\lambda}\cos\sqrt{\lambda}l=0,$$
$$\lambda=\lambda_k=\frac{(2k-1)^2\pi^2}{4l^2},\quad X_k(x)=C_k\sin\frac{(2k-1)\pi}{2l}x,\quad k=1,2,\cdots.$$

$U(x,t)$的通解为
\begin{align*}
  U(x,t)
   & =\sum_{k=1}^\infty\left(A_k\cos\sqrt{\lambda}at+
  B_k\sin\sqrt{\lambda}at\right)\sin\sqrt{\lambda}x          \\
   & =\sum_{k=1}^\infty\left(A_k\cos\frac{(2k-1)\pi a}{2l}t+
  B_k\sin\frac{(2k-1)\pi a}{2l}t\right)\sin\frac{(2k-1)\pi}{2l}x.
\end{align*}

代入初值条件得
$$\varphi(x)=0=\sum_{k=1}^\infty A_k\sin\sqrt{\lambda}x=\sum_{k=1}^\infty A_k\sin\frac{(2k-1)\pi }{2l}x,$$
$$\psi(x)=\sin\frac{\pi x}{2l}=\sum_{k=1}^\infty B_k\sqrt{\lambda}a\sin\sqrt{\lambda}x=\sum_{k=1}^\infty \frac{B_k(2k-1)\pi a}{2l}\sin\frac{(2k-1)\pi}{2l}x.$$

解得
$$A_k=0,\quad B_k=\left\{\begin{aligned}&\frac{2l}{\pi a}, &\quad k = 1 \\& 0, &\quad k\neq 1 \end{aligned}\right..$$

故
$$U(x,t)=\frac{2l}{\pi a}\sin\frac{\pi a}{2l}t\sin\frac{\pi}{2l}x.$$

再求
$$
  \left\{\begin{aligned}
     & V_{tt}-a^2V_{xx}=g, \quad 0<x<l,\quad t>0, \\
     & V|_{x=0}=V_x|_{x=l}=0,                     \\
     & V|_{t=0}=V_t|_{t=0}=0.
  \end{aligned}\right.
$$

令$f(x,\tau)=g$, 代入齐次化原理公式得
\begin{align*}
  B_k(\tau)
   & =\frac{2}{l\sqrt{\lambda}a}\int_0^lf(x,\tau)\sin\sqrt{\lambda}xdx=\frac{4}{(2k-1)\pi a}\int_0^lg\sin\frac{(2k-1)\pi}{2l}xdx \\
   & =-\frac{8gl(\sin k\pi -1)}{(2k-1)^2\pi^2a}=\frac{8gl}{(2k-1)^2\pi^2a}.
\end{align*}

故
\begin{align*}
  V(x,t) & =\sum_{k=1}^\infty\int_0^t B_k(\tau)\sin\sqrt{\lambda}a(t-\tau)d\tau\cdot\sin\sqrt{\lambda}x                                       \\
         & =\sum_{k=1}^\infty\int_0^t \frac{8gl}{(2k-1)^2\pi^2a}\sin\frac{(2k-1)\pi a}{2l}(t-\tau)d\tau\cdot\sin\frac{(2k-1)\pi}{2l}x         \\
         & =\sum_{k=1}^\infty\frac{8gl}{(2k-1)^2\pi^2a}\cdot\frac{4l\sin^2\frac{(2k-1)\pi a}{4l}t}{(2k-1)\pi a}\cdot\sin\frac{(2k-1)\pi}{2l}x \\
         & =\sum_{k=1}^\infty\frac{32gl^2}{(2k-1)^3\pi^3a^2}\sin^2\frac{(2k-1)\pi a}{4l}t\sin\frac{(2k-1)\pi}{2l}x.
\end{align*}
$$u(x,t)=U(x,t)+V(x,t)=\frac{2l}{\pi a}\sin\frac{\pi a}{2l}t\sin\frac{\pi}{2l}x+\sum_{k=1}^\infty\frac{32gl^2}{(2k-1)^3\pi^3a^2}\sin^2\frac{(2k-1)\pi a}{4l}t\sin\frac{(2k-1)\pi}{2l}x.$$

\section{1.3/5}
\begin{problem}
用分离变量法求下面问题的解:
$$
  \left\{\begin{aligned}
     & \frac{\partial^2u}{\partial t^2}=
    a^2\frac{\partial^2u}{\partial x^2}+b\sinh x,                   \\
     & u|_{x=0}=u|_{x=l}=0,                                         \\
     & u|_{t=0}=\left.\frac{\partial u}{\partial t}\right|_{t=0}=0.
  \end{aligned}\right.
$$
\end{problem}

令$f(x,\tau)=b\sinh x$, 代入齐次化原理公式得
\begin{align*}
  B_k(\tau)
   & =\frac{2}{l\sqrt{\lambda}a}\int_0^lf(x,\tau)\sin\sqrt{\lambda}xdx
  =\frac{2}{k\pi a}\int_0^lb\sinh x\sin\frac{k\pi}{l}xdx                       \\
   & =\frac{2bl(l\cosh l\sin k\pi-k\pi\sinh l\cos k\pi)}{k\pi a(k^2\pi^2+l^2)} \\
   & =(-1)^{k+1}\frac{2bl\sinh l}{a(k^2\pi^2+l^2)}.
\end{align*}

故
\begin{align*}
  u(x,t) & =\sum_{k=1}^\infty\int_0^t B_k(\tau)\sin\sqrt{\lambda}a(t-\tau)d\tau\cdot\sin\sqrt{\lambda}x                                          \\
         & =\sum_{k=1}^\infty\int_0^t (-1)^{k+1}\frac{2bl\sinh l}{a(k^2\pi^2+l^2)}\sin\frac{k\pi}{l}a(t-\tau)d\tau\cdot\sin\frac{k\pi}{l}x       \\
         & =\sum_{k=1}^\infty (-1)^{k+1}\frac{2bl\sinh l}{a(k^2\pi^2+l^2)}\cdot\frac{2l\sin^2\frac{k\pi a}{2l}t}{k\pi a}\cdot\sin\frac{k\pi}{l}x \\
         & =\sum_{k=1}^\infty (-1)^{k+1}\frac{2bl^2\sinh l}{k\pi a^2(k^2\pi^2+l^2)}\sin^2\frac{k\pi a}{2l}t\sin\frac{k\pi}{l}x.
\end{align*}

\section{1.3/6}
\begin{problem}
用分离变量法求下面问题的解:
$$
  \left\{\begin{aligned}
     & \frac{\partial^2u}{\partial t^2}+2b\frac{\partial u}{\partial t}=
    a^2\frac{\partial^2u}{\partial x^2},                                 \\
     & u|_{x=0}=u|_{x=l}=0,                                              \\
     & u|_{t=0}=\frac{h}{l}x,
    \quad \left.\frac{\partial u}{\partial t}\right|_{t=0}=0.
  \end{aligned}\right.
$$
\end{problem}

设$u(x,t)=X(x)T(t)$, 代入得
$$X(x)T''(t)+2bX(x)T'(t)=a^2X''(x)T(t),$$
$$\frac{T''(t)+2bT'(t)}{a^2T(t)}=\frac{X''(x)}{X(x)}=-\lambda.$$

则有两个常微分方程
$$T''(t)+2bT'(t)+\lambda a^2T(t)=0,$$
$$X''(x)+\lambda X(x)=0.$$

方程的特征值和对应的特征函数为
$$\lambda=\lambda_k=\frac{k^2\pi^2}{l^2},\quad X_k(x)=C_k\sin\sqrt{\lambda}x=C_k\sin\frac{k\pi}{l}x,\quad k=1,2,\cdots.$$

代入关于$T$的常微分方程可得
$$T''(t)+2bT'(t)+\frac{k^2\pi^2a^2}{l^2}T(t)=0,$$
$$\Delta=4b^2-\frac{4k^2\pi^2a^2}{l^2},$$

设$$\lambda'=\frac{k^2\pi^2a^2}{l^2}-b^2,$$
$$r_1=-b+\sqrt{b^2-\frac{k^2\pi^2a^2}{l^2}}=-b+\sqrt{-\lambda'},\quad r_2=-b-\sqrt{b^2-\frac{k^2\pi^2a^2}{l^2}}=-b-\sqrt{-\lambda'}.$$

当$\lambda'<0$时
$$k<\frac{bl}{\pi a},\quad T(t)=C_1e^{r_1t}+C_2e^{r_2t},$$
$$U_k(x,t)=T_k(t)X_k(x)=(A_ke^{r_1t}+B_ke^{r_2t})\sin\sqrt{\lambda}x,$$
$$\frac{\partial U_k}{\partial t}(x,t)=T_k'(t)X_k(x)=(A_kr_1e^{r_1t}+B_kr_2e^{r_2t})\sin\sqrt{\lambda}x.$$

当$\lambda'=0$时
$$k=\frac{bl}{\pi a},\quad T(t)=e^{-bt}(C_1+C_2t),$$
$$U_k(x,t)=T_k(t)X_k(x)=e^{-bt}(A_k+B_kt)\sin\sqrt{\lambda}x,$$
$$\frac{\partial U_k}{\partial t}(x,t)=T_k'(t)X_k(x)=e^{-bt}(-bA_k+B_k-bB_kt)\sin\sqrt{\lambda}x.$$

当$\lambda'>0$时
$$k>\frac{bl}{\pi a},\quad T(t)=e^{-bt}(C_1\cos\sqrt{\lambda'}t+C_2\sin\sqrt{\lambda'}t),$$
$$U_k(x,t)=T_k(t)X_k(x)=e^{-bt}(A_k\cos\sqrt{\lambda'}t+B_k\sin\sqrt{\lambda'}t)\sin\sqrt{\lambda}x,$$
$$\frac{\partial U_k}{\partial t}(x,t)=T_k'(t)X_k(x)=e^{-bt}(-bA_k\cos\sqrt{\lambda'}t-bB_k\sin\sqrt{\lambda'}t-\sqrt{\lambda'}A_k\sin\sqrt{\lambda'}t+\sqrt{\lambda'}B_k\sin\sqrt{\lambda'}t)\sin\sqrt{\lambda}x.$$

设
$$n=\frac{bl}{\pi a},\quad \underline{n}=\lceil n-1\rceil,\quad \overline{n}=\lfloor n+1\rfloor,\quad \delta(n)=\left\{\begin{aligned}1,&\quad n\in Z\\0,&\quad n\not\in Z\end{aligned}\right..$$

方程的通解为
\begin{align*}
  u(x,t) & =\sum_{k=1}^\infty T_k(t)X_k(x)                                                                                                                                                                        \\
         & =\left[\sum_{k=1}^{\underline{n}}(A_ke^{r_1t}+B_ke^{r_2t})+\delta(n)e^{-bt}(A_n+B_nt)+\sum_{k=\overline{n}}^\infty e^{-bt}(A_k\cos\sqrt{\lambda'}t+B_k\sin\sqrt{\lambda'}t)\right]\sin\sqrt{\lambda}x.
\end{align*}

代入初值条件得
$$u(x,0)=\left[\sum_{k=1}^{\underline{n}}(A_k+B_k)+\delta(n)A_n+\sum_{k=\overline{n}}^\infty A_k\right]\sin\sqrt{\lambda}x=\frac{h}{l}x,$$
$$\frac{\partial u}{\partial t}(x,0)=\left[\sum_{k=1}^{\underline{n}}(A_kr_1+B_kr_2)+\delta(n)(-A_n+B_n)+\sum_{k=\overline{n}}^\infty(-bA_k+\sqrt{\lambda'}B_k)\right]\sin\sqrt{\lambda}x=0.$$

解得
$$A_k'=\left\{\begin{aligned}&A_k+B_k,&\quad k<n\\&A_k,&\quad k\geqslant n\end{aligned}\right.
  \quad=\frac{2}{l}\int_0^l\frac{h}{l}x\sin\sqrt{\lambda}xdx=\frac{h}{l}\cdot-\frac{2(\sqrt{\lambda}l\cos\sqrt{\lambda}l-\sin\sqrt{\lambda}l)}{\lambda l}=(-1)^{k+1}\frac{2h}{k\pi},$$
$$B_k'=\left\{\begin{aligned}&A_kr_1+B_kr_2,&\quad k<n\\&-A_k+B_k,&\quad k=n\\&-bA_k+\sqrt{\lambda'}B_k,&\quad k>n\end{aligned}\right.\quad=0.$$

化简得
$$A_k=\left\{\begin{aligned}&\frac{r_2}{r_2-r_1}A_k',&\quad k<n\\&A_k',&\quad k\geqslant n\end{aligned}\right.,\quad
  B_k=\left\{\begin{aligned}&-\frac{r_1}{r_2-r_1}A_k',&\quad k<n\\&A_k',&\quad k=n\\&\frac{b}{\sqrt{\lambda'}}A_k',&\quad k>n\end{aligned}\right..$$

故将$r_1,r_2,\lambda,\lambda',n,\underline{n},\overline{n},\delta(n),A_k,B_k$代入可得$u(x,t)$.
% $$u(x,t)=\left[\sum_{k=1}^{\underline{n}}(A_ke^{r_1t}+B_ke^{r_2t})+\delta(n)e^{-bt}(A_n+B_nt)+\sum_{k=\overline{n}}^\infty e^{-bt}(A_k\cos\sqrt{\lambda'}t+B_k\sin\sqrt{\lambda'}t)\right]\sin\sqrt{\lambda}x,$$

\section*{例题}

\subsection*{(1)}

\begin{problem}
$$
  \left\{\begin{aligned}
     & u_{tt}-a^2u_{xx}=A\sin\omega t, \\
     & u|_{x=0}=u_x|_{x=l}=0,          \\
     & u|_{t=0}=1,\quad u_t|_{t=0}=x.
  \end{aligned}\right.
$$
\end{problem}

利用叠加原理, 设$$u(x,t)=U(x,t)+V(x,t).$$

先求
$$
  \left\{\begin{aligned}
     & U_{tt}-a^2U_{xx}=0,            \\
     & U|_{x=0}=U_x|_{x=l}=0,         \\
     & U|_{t=0}=1,\quad U_t|_{t=0}=x.
  \end{aligned}\right.
$$

方程的特征值和对应的特征函数为
$$X''(x)+\lambda X(x)=0,$$
$$X(x)=C_1\cos\sqrt{\lambda}x+C_2\sin\sqrt{\lambda}x,$$

代入$X(0)=0$, $X'(l)=0$得
$$C_1=0,\quad C_2\sqrt{\lambda}\cos\sqrt{\lambda}l=0,$$
$$\lambda=\lambda_k=\frac{(2k-1)^2\pi^2}{4l^2},\quad X_k(x)=C_k\sin\frac{(2k-1)\pi}{2l}x,\quad k=1,2,\cdots.$$

$U(x,t)$的通解为
\begin{align*}
  U(x,t)
   & =\sum_{k=1}^\infty\left(A_k\cos\sqrt{\lambda}at+
  B_k\sin\sqrt{\lambda}at\right)\sin\sqrt{\lambda}x          \\
   & =\sum_{k=1}^\infty\left(A_k\cos\frac{(2k-1)\pi a}{2l}t+
  B_k\sin\frac{(2k-1)\pi a}{2l}t\right)\sin\frac{(2k-1)\pi}{2l}x.
\end{align*}

代入初值条件得
$$\varphi(x)=1=\sum_{k=1}^\infty A_k\sin\sqrt{\lambda}x=\sum_{k=1}^\infty A_k\sin\frac{(2k-1)\pi }{2l}x,$$
$$\psi(x)=x=\sum_{k=1}^\infty B_k\sqrt{\lambda}a\sin\sqrt{\lambda}x=\sum_{k=1}^\infty \frac{B_k(2k-1)\pi a}{2l}\sin\frac{(2k-1)\pi}{2l}x.$$

解得
$$A_k=\frac{2}{l}\int_0^l\varphi(x)\sin\sqrt{\lambda}xdx=\frac{2}{l}\int_0^l\sin\frac{(2k-1)\pi}{2l}xdx=\frac{4(-1+\sin k\pi)}{(2k-1)\pi}=-\frac{4}{(2k-1)\pi},$$
$$B_k=\frac{2}{l\sqrt{\lambda}a}\int_0^l\psi(x)\sin\sqrt{\lambda}xdx=
  \frac{4}{(2k-1)\pi a}\int_0^l x\sin\frac{(2k-1)\pi}{2l}xdx=-\frac{4l^2[2\cos k\pi+(2k-1)\pi\sin k\pi]}{(2k-1)^3\pi^3a}.$$

化简得
$$B_k=(-1)^{k+1}\frac{8l^2}{(2k-1)^3\pi^3a}.$$

故
$$U(x,t)=\sum_{k=1}^\infty\left[-\frac{4}{(2k-1)\pi}\cos\frac{(2k-1)\pi a}{2l}t+
    (-1)^{k+1}\frac{8l^2}{(2k-1)^3\pi^3a}\sin\frac{(2k-1)\pi a}{2l}t\right]\sin\frac{(2k-1)\pi}{2l}x.$$

再求
$$
  \left\{\begin{aligned}
     & V_{tt}-a^2V_{xx}=A\sin\omega t, \\
     & V|_{x=0}=V_x|_{x=l}=0,          \\
     & V|_{t=0}=V_t|_{t=0}=0.
  \end{aligned}\right.
$$

令$f(x,\tau)=A\sin\omega t$, 代入齐次化原理公式得
\begin{align*}
  B_k(\tau)
   & =\frac{2}{l\sqrt{\lambda}a}\int_0^lf(x,\tau)\sin\sqrt{\lambda}xdx=\frac{4}{(2k-1)\pi a}\int_0^lA\sin\omega \tau\sin\frac{(2k-1)\pi}{2l}xdx \\
   & =-\frac{8A\sin\omega \tau\cdot l(\sin k\pi -1)}{(2k-1)^2\pi^2a}=\frac{8lA\sin\omega \tau}{(2k-1)^2\pi^2a}.
\end{align*}

故
\begin{align*}
  V(x,t) & =\sum_{k=1}^\infty\int_0^t B_k(\tau)\sin\sqrt{\lambda}a(t-\tau)d\tau\cdot\sin\sqrt{\lambda}x                                                                                            \\
         & =\sum_{k=1}^\infty\int_0^t \frac{8lA\sin\omega \tau}{(2k-1)^2\pi^2a}\sin\frac{(2k-1)\pi a}{2l}(t-\tau)d\tau\cdot\sin\frac{(2k-1)\pi}{2l}x                                               \\
         & =\sum_{k=1}^\infty\frac{8lA}{(2k-1)^2\pi^2a}\cdot\frac{-2l[2\omega l\sin\frac{(2k-1)\pi a}{2l}t+(2k-1)\pi a\sin\omega t]}{(2k-1)^2\pi^2 a^2-4\omega^2l^2}\cdot\sin\frac{(2k-1)\pi}{2l}x \\
         & =\sum_{k=1}^\infty-\frac{16l^2A}{(2k-1)^4\pi^4a^3-4(2k-1)^2\pi^2a\omega^2l^2}\left[2\omega l\sin\frac{(2k-1)\pi a}{2l}t+(2k-1)\pi a\sin\omega t\right]\sin\frac{(2k-1)\pi}{2l}x.
\end{align*}
$$u(x,t)=U(x,t)+V(x,t).$$

\subsection*{(2)}
\begin{problem}
$$
  \left\{\begin{aligned}
     & u_{tt}-u_{xx}-4u=2\sin^2x,     \\
     & u_x|_{x=0}=u_x|_{x=\pi}=0,     \\
     & u|_{t=0}=0,\quad u_t|_{t=0}=0.
  \end{aligned}\right.
$$
\end{problem}


将原方程化为
$$
  \left\{\begin{aligned}
     & t'=t-\tau                                  \\
     & W_{tt}-W_{xx}-4W=0,                        \\
     & W_x|_{x=0}=W_x|_{x=\pi}=0,                 \\
     & W|_{t'=0}=0,\quad W_{t'}|_{t'=0}=2\sin^2x.
  \end{aligned}\right.
$$

设$W(x,t)=X(x)T(t)$, 代入得
$$X(x)T''(t)-X''(x)T(t)-4X(x)T(t)=0,$$
$$\frac{T''(t)-4T(t)}{T(t)}=\frac{X''(x)}{X(x)}=-\lambda.$$

则有两个常微分方程
$$T''(t)+(\lambda-4)T(t)=0,$$
$$X''(x)+\lambda X(x)=0.$$

方程的特征值和对应的特征函数为
$$X(x)=C_1\cos\sqrt{\lambda}x+C_2\sin\sqrt{\lambda}x,$$

代入$X'(0)=0$, $X'(\pi)=0$得
$$C_2=0,\quad -C_1\sqrt{\lambda}\sin\sqrt{\lambda}\pi=0,$$
$$\lambda=\lambda_k=k^2,\quad X_k(x)=C_k\cos\sqrt{\lambda}x=C_k\cos kx,\quad k=0,1,2,\cdots.$$

代入关于$T$的常微分方程可得
$$T''(t)+(k^2-4)T(t)=0,$$
% $$\Delta=4b^2-\frac{4k^2\pi^2a^2}{l^2},$$

设$$\lambda'=k^2-4,$$
$$r_1=\sqrt{4-k^2}=\sqrt{-\lambda'},\quad r_2=-\sqrt{4-k^2}=-\sqrt{-\lambda'}.$$

当$\lambda'<0$时
$$k<2,\quad T(t)=C_1e^{\sqrt{-\lambda'}t}+C_2e^{-\sqrt{-\lambda'}t},$$
$$W_k(x,t)=T_k(t)X_k(x)=(A_ke^{\sqrt{-\lambda'}t}+B_ke^{-\sqrt{-\lambda'}t})\cos\sqrt{\lambda}x,$$
$$\frac{\partial W_k}{\partial t}(x,t)=T_k'(t)X_k(x)=\sqrt{-\lambda'}(A_ke^{\sqrt{-\lambda'}t}-B_ke^{-\sqrt{-\lambda'}t})\cos\sqrt{\lambda}x.$$

当$\lambda'=0$时
$$k=2,\quad T(t)=C_1+C_2t,$$
$$W_k(x,t)=T_k(t)X_k(x)=(A_k+B_kt)\cos\sqrt{\lambda}x,$$
$$\frac{\partial W_k}{\partial t}(x,t)=T_k'(t)X_k(x)=B_k\cos\sqrt{\lambda}x.$$

当$\lambda'>0$时
$$k>2,\quad T(t)=C_1\cos\sqrt{\lambda'}t+C_2\sin\sqrt{\lambda'}t,$$
$$W_k(x,t)=T_k(t)X_k(x)=(A_k\cos\sqrt{\lambda'}t+B_k\sin\sqrt{\lambda'}t)\cos\sqrt{\lambda}x,$$
$$\frac{\partial W_k}{\partial t}(x,t)=T_k'(t)X_k(x)=\sqrt{\lambda'}(-A_ksin\sqrt{\lambda'}t+B_k\cos\sqrt{\lambda'}t)\cos\sqrt{\lambda}x.$$

方程的通解为
\begin{align*}
  W(x,t)
   & =\sum_{k=0}^\infty T_k(t)X_k(x)              \\
   & =\left[\sum_{k=0}^1(A_ke^{\sqrt{-\lambda'}t}
  +B_ke^{-\sqrt{-\lambda'}t})+(A_2+B_2t)
  +\sum_{k=3}^\infty(A_k\cos\sqrt{\lambda'}t
  +B_k\sin\sqrt{\lambda'}t)\right]\cos\sqrt{\lambda}x.
\end{align*}

代入初值条件得
$$W(x,0)=\left[\sum_{k=0}^1(A_k+B_k)+A_2+\sum_{k=3}^\infty A_k\right]\cos\sqrt{\lambda}x=0,$$
$$\frac{\partial W}{\partial t}(x,0)=\left[\sum_{k=0}^1\sqrt{-\lambda'}(A_k-B_k)+B_2+\sum_{k=3}^\infty\sqrt{\lambda'}B_k\right]\cos\sqrt{\lambda}x=2\sin^2x=1-\cos2x.$$

解得
$$A_k'=\left\{\begin{aligned}&A_k+B_k,&\quad k<2\\&A_k,&\quad k\geqslant 2\end{aligned}\right.\quad=0,$$
$$B_k'=\left\{\begin{aligned}&\sqrt{-\lambda'}(A_k-B_k),&\quad k<2\\&B_k,&\quad k=2\\&\sqrt{\lambda'}B_k,&\quad k>2\end{aligned}\right.\quad=\left\{\begin{aligned}&1,&\quad k=0\\&-1,&\quad k=2\\&0,&\quad k\neq 0,2\end{aligned}\right..$$

化简得
$$A_k=\left\{\begin{aligned}&\frac{1}{4},&\quad k=0\\&0,&\quad k\neq 0\end{aligned}\right.,\quad
  B_k=\left\{\begin{aligned}&-\frac{1}{4},&\quad k=0\\&-1,&\quad k=2\\&0,&\quad k\neq0,2\end{aligned}\right..$$

故
$$W(x,t)=\frac{1}{4}(e^{2t}-e^{-2t})-t\cos2x,$$
$$W(x,t;\tau)=\frac{1}{4}(e^{2(t-\tau)}-e^{-2(t-\tau)})-(t-\tau)\cos2x,$$
$$u(x,t)=\int_0^tW(x,t;\tau)d\tau =\int_0^t\frac{1}{4}(e^{2(t-\tau)}-e^{-2(t-\tau)})-(t-\tau)\cos2xd\tau =\frac{1}{4}\cosh 2t-\frac{1}{2}t^2\cos 2x-\frac{1}{4}.$$


\end{document}
