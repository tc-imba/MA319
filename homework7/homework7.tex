\documentclass[11pt,a4paper]{article}
\usepackage{../ma319}
\semester{Fall}
\year{2019}
\subtitlenumber{7}
\author{刘逸灏 (515370910207)}

\begin{document}

\maketitle

\section{2.2/3}
\begin{problem}
如果有一长度为$l$的均匀细棒, 其周围及两端$x=0$, $x=l$均为绝热, 初始温度分布为$u(x,0)=f(x)$, 问以后时刻的温度分布如何? 且证明当$f(x)$等于常数$u_0$时, 恒有$u(x,t)=u_0$.
\end{problem}

题目可归结为求解如下的定解问题:
$$\left\{\begin{aligned}
     & \frac{\partial u}{\partial t}=a^2\frac{\partial^2u}{\partial x^2},        \\
     & \frac{\partial u}{\partial x}(0,t) =\frac{\partial u}{\partial x}(l,t)=0, \\
     & u(x,0)=f(x).
  \end{aligned}\right.$$

方程的特征值和对应的特征函数为
$$X''(x)+\lambda X(x)=0,$$
$$X(x)=C_1\cos\sqrt{\lambda}x+C_2\sin\sqrt{\lambda}x,$$

代入$X'(0)=0$, $X'(l)=0$得
$$C_2=0,\quad C_1\sqrt{\lambda}\sin\sqrt{\lambda}l=0,$$
$$\lambda=\lambda_k=\frac{k^2\pi^2}{l^2},\quad X_k(x)=C_k\cos\frac{k\pi}{l}x,\quad k=0,1,2,\cdots.$$

方程的通解为
$$u(x,t)=\frac{1}{2}A_0+\sum_{k=1}^\infty A_ke^{-a^2\lambda t}\cos\sqrt{\lambda}x=\frac{1}{2}A_0+\sum_{k=1}^\infty A_k e^{-\frac{k^2\pi^2}{l^2}a^2 t}\cos\frac{k\pi}{l}x.$$

代入初值条件得
$$u(x,0)=\frac{1}{2}A_0+\sum_{k=1}^\infty A_k\cos\sqrt{\lambda}x=\frac{1}{2}A_0+\sum_{k=1}^\infty A_k\cos\frac{k\pi}{l}x=f(x).$$

解得
$$A_k=\frac{2}{l}\int_0^l f(x)\cos\sqrt{\lambda}xdx=\frac{2}{l}\int_0^l f(x)\cos\frac{k\pi}{l}xdx.$$

故
$$u(x,t)=\frac{1}{l}\int_0^l f(x)dx+\sum_{k=1}^\infty \frac{2}{l}\int_0^l f(x)\cos\frac{k\pi}{l}xdx \cdot e^{-\frac{k^2\pi^2}{l^2}a^2 t}\cos\frac{k\pi}{l}x.$$

当$f(x)$等于常数$u_0$时
$$A_k=\left\{\begin{aligned}
     & 2u_0,\quad                         & k=0, \\
     & \frac{u_0l\sin k\pi}{k\pi}=0,\quad & k>0.
  \end{aligned}\right.$$

故
$$u(x,t)=\frac{1}{2}A_0=u_0.$$

\section{2.2/4}
\begin{problem}
在区域$t>0$, $0<x<l$中求解如下的定解问题:
$$\left\{\begin{aligned}
     & \frac{\partial u}{\partial t}=\frac{\partial^2u}{\partial x^2}-\beta(u-u_0), \\
     & u(0,t) = u(1,t)=u_0,                                                         \\
     & u(x,0)=f(x),
  \end{aligned}\right.$$
其中$a$, $\beta$, $u_0$均为常数, $f(x)$为已知函数.
\end{problem}

设
$$u(x,t)=u_0+v(x,t)e^{-\beta t}.$$

则
$$\frac{\partial u}{\partial t}-\frac{\partial^2u}{\partial x^2}+\beta(u-u_0)=\frac{\partial v}{\partial t}e^{-\beta t}-\beta ve^{-\beta t}-\frac{\partial^2v}{\partial x^2}e^{-\beta t}+\beta ve^{-\beta t}=e^{-\beta t}\left(\frac{\partial v}{\partial t}-\frac{\partial^2v}{\partial x^2}\right)=0,$$
$$v(0,t)e^{-\beta t}=v(l,t)e^{-\beta t}=u(0,t)-u_0=u(l,t)-u_0=0,$$
$$v(x,0)=u(x,0)-u_0=f(x)-u_0.$$

故可以先求解如下定解问题:
$$\left\{\begin{aligned}
     & \frac{\partial v}{\partial t}=\frac{\partial^2v}{\partial x^2}, \\
     & v(0,t) = v(1,t)=0,                                              \\
     & v(x,0)=f(x)-u_0.
  \end{aligned}\right.$$

方程的特征值和对应的特征函数为
$$\lambda=\lambda_k=\frac{k^2\pi^2}{l^2},\quad X_k(x)=C_k\sin\sqrt{\lambda}x=C_k\sin\frac{k\pi}{l} x,\quad k=1,2,\cdots.$$

方程的通解为
$$v(x,t)=\sum_{k=1}^\infty A_ke^{-a^2\lambda t}\sin\sqrt{\lambda}x=\sum_{k=1}^\infty A_ke^{-\frac{k^2\pi^2}{l^2}a^2t}\sin \frac{k\pi}{l} x.$$

代入初值条件得
$$v(x,0)=\sum_{k=1}^\infty A_k\sin\sqrt{\lambda}x=\sum_{k=1}^\infty A_k\sin \frac{k\pi}{l} x=f(x)-u_0.$$

解得
$$A_k=\frac{2}{l}\int_0^l [f(x)-u_0]\sin\sqrt{\lambda}xdx=\frac{2}{l}\int_0^l [f(x)-u_0]\sin \frac{k\pi}{l} xdx
  =\frac{2}{l}\int_0^l f(x)\sin \frac{k\pi}{l} xdx-\frac{2u_0(1-\cos k\pi)}{k\pi}.$$

化简得
$$A_k=\frac{2}{l}\int_0^l f(x)\sin \frac{k\pi}{l} xdx+\frac{2u_0[(-1)^k-1]}{k\pi}.$$

故
$$v(x,t)=\sum_{k=1}^\infty \left\{\frac{2}{l}\int_0^l f(x)\sin \frac{k\pi}{l} xdx+\frac{2u_0[(-1)^k-1]}{k\pi}\right\} e^{-\frac{k^2\pi^2}{l^2}a^2t}\sin \frac{k\pi}{l} x,$$
$$u(x,t)=u_0+e^{-\beta t}\sum_{k=1}^\infty \left\{\frac{2}{l}\int_0^l f(x)\sin \frac{k\pi}{l} xdx+\frac{2u_0[(-1)^k-1]}{k\pi}\right\} e^{-\frac{k^2\pi^2}{l^2}a^2t}\sin \frac{k\pi}{l} x.$$

\section{2.2/5}
\begin{problem}
长度为$l$的均匀细杆的初始温度为$0^\circ C$, 端点$x=0$保持常温$u_0$, 而在$x=l$和侧面上, 热量可以发散到周围的介质中去, 介质的温度为$^\circ C$, 此时杆上的温度分布函数$u(x,t)$满足下述定解问题:
$$\left\{\begin{aligned}
     & \frac{\partial u}{\partial t}=a^2\frac{\partial^2u}{\partial x^2}-b^2u,              \\
     & u(0,t)=u_0,\quad \left.\left(\frac{\partial u}{\partial x}+Hu\right)\right|_{x=l}=0, \\
     & u(x,0)=0,
  \end{aligned}\right.$$
其中$a,b,H$均为常数, 试求出$u(x,t)$.
\end{problem}

设
$$u(x,t)=v(x,t)e^{-b^2t}+f(x).$$

则
$$\frac{\partial u}{\partial t}-a^2\frac{\partial^2u}{\partial x^2}+b^2u=\frac{\partial v}{\partial t}e^{-b^2t}-b^2ve^{-b^2t}-a^2\frac{\partial^2v}{\partial x^2}e^{-b^2t}-a^2f''(x)+b^2ve^{-b^2t}+b^2f(x)=0,$$
$$v(0,t)e^{-b^2t}+f(0)=u_0,\quad \left.\left(\frac{\partial v}{\partial x}+Hv\right)\right|_{x=l}+f'(l)+Hf(l)=0,$$
$$v(x,0)+f(x)=0.$$

可找到$f(x)$使以下成立
$$\left(\frac{\partial v}{\partial t}-\frac{\partial^2v}{\partial x^2}\right)e^{-b^2t}=0,\quad f''(x)-\frac{b^2}{a^2}f(x)=0,$$
$$v(0,t)e^{-b^2t}=0,\quad f(0)=u_0,\quad \left.\left(\frac{\partial v}{\partial x}+Hv\right)\right|_{x=l}=0,\quad f'(l)+Hf(l)=0.$$

故$f(x)$的通解为
$$f(x)=C_1e^{\frac{b}{a}x}+C_2e^{-\frac{b}{a}x}.$$
代入初值条件得
$$f(0)=C_1+C_2=u_0,$$
$$f'(l)+Hf(l)=\frac{b}{a}\left(C_1e^{\frac{bl}{a}}-C_2e^{-\frac{bl}{a}}\right)+H\left(C_1e^{\frac{bl}{a}}+C_2e^{-\frac{bl}{a}}\right)=0.$$
解得
$$C_1=\frac{u_0(b-a H)}{a H e^{\frac{2 b l}{a}}+b e^{\frac{2 b l}{a}}-a H+b},\quad C_2=\frac{u_0(a H+b) e^{\frac{2 b l}{a}}}{a H e^{\frac{2 b l}{a}}+b e^{\frac{2 b l}{a}}-a H+b},$$
$$f(x)=\frac{u_0 e^{-\frac{b x}{a}} \left[a H \left(e^{\frac{2 b l}{a}}-e^{\frac{2 b x}{a}}\right)+b \left(e^{\frac{2 b l}{a}}+e^{\frac{2 b x}{a}}\right)\right]}{a H \left(e^{\frac{2 b l}{a}}-1\right)+b \left(e^{\frac{2 b l}{a}}+1\right)}=\frac{u_0 \left[a H \sinh \left(\frac{b (l-x)}{a}\right)+b \cosh \left(\frac{b (l-x)}{a}\right)\right]}{a H \sinh \left(\frac{b l}{a}\right)+b \cosh \left(\frac{b l}{a}\right)}.$$

故可以先求解如下定解问题:
$$\left\{\begin{aligned}
     & \frac{\partial v}{\partial t}=a^2\frac{\partial^2v}{\partial x^2},                 \\
     & v(0,t)=0,\quad \left.\left(\frac{\partial v}{\partial x}+Hv\right)\right|_{x=l}=0, \\
     & v(x,0)=-f(x).
  \end{aligned}\right.$$

方程的特征值和对应的特征函数为
$$X''(x)+\lambda X(x)=0,$$
$$X(x)=C_1\cos\sqrt{\lambda}x+C_2\sin\sqrt{\lambda}x,$$

代入$X(0)=0$, $X'(l)+HX(l)=0$得
$$C_1=0,\quad C_2(\sqrt{\lambda}\cos\sqrt{\lambda}l+H\sin\sqrt{\lambda}l)=0,$$
$$v=\sqrt{\lambda}l,\quad \tan v=-\frac{\sqrt{\lambda}}{H}.$$
存在无数个正根$v_k>0$, 满足
$$\left(k-\frac{1}{2}\right)\pi<v_k<k\pi,\quad \lambda_k=\left(\frac{v_k}{l}\right)^2,\quad X_k(x)=C_k\sin\frac{v_k}{l}x,\quad k=1,2,\cdots$$

方程的通解为
$$v(x,t)=\sum_{k=1}^\infty A_ke^{-a^2\lambda t}\sin\sqrt{\lambda}x=\sum_{k=1}^\infty A_ke^{-\frac{v_k^2}{l^2}a^2t}\sin \frac{v_k}{l} x.$$

代入初值条件得
$$v(x,0)=\sum_{k=1}^\infty A_k\sin\sqrt{\lambda}x=\sum_{k=1}^\infty A_k\sin \frac{v_k}{l} x=-f(x).$$

由固有函数系$\{X_k\}=\{\sin\sqrt{\lambda_k}x\}$的正交性可得
$$M_k=\int_0^l\sin^2\sqrt{\lambda_k}xdx=\frac{l}{2}+\frac{H}{2(H^2+\lambda_k)}=\frac{l}{2}+\frac{Hl^2}{2(H^2l^2+v_k^2)},$$
$$A_k=\frac{1}{M_k}\int_0^l -f(x)\sin\sqrt{\lambda_k}xdx=-\frac{1}{M_k}\int_0^l f(x)\sin\frac{v_k}{l}xdx.$$

故
$$v(x,t)=-\sum_{k=1}^\infty \frac{1}{M_k}\int_0^l f(\xi)\sin\frac{v_k}{l}\xi d\xi \cdot e^{-\frac{v_k^2}{l^2}a^2t}\sin \frac{v_k}{l} x,$$
$$u(x,t)=-e^{-b^2t}\sum_{k=1}^\infty \frac{1}{M_k}\int_0^l f(\xi)\sin\frac{v_k}{l}\xi d\xi \cdot e^{-\frac{v_k^2}{l^2}a^2t}\sin \frac{v_k}{l} x+\frac{u_0 \left[a H \sinh \left(\frac{b (l-x)}{a}\right)+b \cosh \left(\frac{b (l-x)}{a}\right)\right]}{a H \sinh \left(\frac{b l}{a}\right)+b \cosh \left(\frac{b l}{a}\right)}.$$

\section{2.3/1}
\begin{problem}
求下列函数的傅里叶变换:
\begin{enumerate}
  \item $e^{-\eta x^2}\quad (\eta>0)$;
  \item $e^{-a|x|}\quad (a>0)$.
\end{enumerate}
\end{problem}

\subsection*{(1)}
$$g(\lambda)=\int_{-\infty}^{\infty}f(\xi)e^{-i\lambda\xi}d\xi=\int_{-\infty}^{\infty}e^{-\eta\xi^2-i\lambda\xi}d\xi=\int_{-\infty}^{\infty}e^{-\eta\xi^2-i\lambda\xi+\frac{\lambda^2}{4\eta}-\frac{\lambda^2}{4\eta}}d\xi=e^{-\frac{\lambda^2}{4\eta}}\int_{-\infty}^{\infty}e^{-\eta\left(\xi+\frac{i\lambda}{2\eta}\right)^2}d\xi.$$
设$$\int_{-\infty}^\infty e^{-\eta x^2}dx=\int_{-\infty}^\infty e^{-\eta y^2}dy=A,$$
$$A^2=\int_{-\infty}^\infty e^{-\eta x^2}dx\int_{-\infty}^\infty e^{-\eta y^2}dy=\int_{-\infty}^\infty\int_{-\infty}^\infty e^{-\eta(x^2+y^2)^2}dxdy.$$
令$r^2=x^2+y^2$
$$A^2=\int_0^{2\pi}\int_0^\infty e^{-\eta r^2}rdrd\theta=2\pi\int_0^\infty-\frac{1}{2\eta}e^{-\eta r^2}d(-\eta r^2)=\frac{\pi}{\eta}\int_{-\infty}^0 e^sds=\frac{\pi}{\eta},$$
$$g(\lambda)=e^{-\frac{\lambda^2}{4\eta}}\int_{-\infty}^{\infty}e^{-\eta\left(\xi+\frac{i\lambda}{2\eta}\right)^2}d\left(\xi+\frac{i\lambda}{2\eta}\right)=e^{-\frac{\lambda^2}{4\eta}}\int_{-\infty}^{\infty}e^{-\eta x^2}dx=Ae^{-\frac{\lambda^2}{4\eta}}=\sqrt{\frac{\pi}{\eta}}e^{-\frac{\lambda^2}{4\eta}}.$$

\subsection*{(2)}
$$g(\lambda)=\int_{-\infty}^{\infty}f(\xi)e^{-i\lambda\xi}d\xi=\int_{-\infty}^{\infty}e^{-a|\xi|-i\lambda\xi}d\xi=\int_{-\infty}^{\infty}e^{-a|\xi|}\cos\lambda\xi d\xi+\int_{-\infty}^{\infty}e^{-a|\xi|}i\sin\lambda\xi d\xi.$$
由于$e^{-a|\xi|}$, $\cos\lambda\xi$是偶函数, $i\sin\lambda\xi$是奇函数,
$$\int_{-\infty}^{\infty}e^{-a|\xi|}\cos\lambda\xi d\xi=2\int_0^\infty e^{-a\xi}\cos\lambda\xi d\xi,\quad \int_{-\infty}^{\infty}e^{-a|\xi|}i\sin\lambda\xi d\xi=0,$$
$$g(\lambda)=2\int_0^\infty e^{-a\xi}\cos\lambda\xi d\xi=\left.\frac{e^{-a \xi } (\lambda  \sin (\lambda  \xi )-a \cos (\lambda  \xi ))}{a^2+\lambda ^2}\right|_0^{\infty}=\frac{a}{a^2+\lambda^2}.$$

\section{2.3/2}
\begin{problem}
证明: 当$f(x)$在$(-\infty,\infty)$上绝对可积时, $F[f]$为连续函数.
\end{problem}

$$F[f](\lambda+h)-F[f](\lambda)=\int_{-\infty}^{\infty}f(\xi)e^{-i\lambda\xi}d\xi-\int_{-\infty}^{\infty}f(\xi)e^{-i(\lambda+h)\xi}d\xi=\int_{-\infty}^{\infty}f(\xi)e^{-i\lambda\xi}(1-e^{-ih\xi})d\xi,$$
$$|F[f](\lambda+h)-F[f](\lambda)|=\left|\int_{-\infty}^{\infty}f(\xi)e^{-i\lambda\xi}(1-e^{-ih\xi})d\xi\right|\leqslant\int_{-\infty}^{\infty}|f(\xi)|\left|e^{-i\lambda\xi}\right|\left|1-e^{-ih\xi}\right|d\xi.$$
易知当$\xi\in\mathbf{R}$时,
$$\left|e^{-i\lambda\xi}\right|=\sqrt{\cos^2\lambda\xi+\sin^2\lambda\xi}=1,\quad \left|1-e^{-ih\xi}\right|=\sqrt{(1-\cos h\xi)^2+\sin^2h\xi}=\sqrt{2}\cdot\sqrt{1-\cos h\xi},$$
$$|F[f](\lambda+h)-F[f](\lambda)|\leqslant\int_{-\infty}^{\infty}|f(\xi)|\sqrt{2}\cdot\sqrt{1-\cos h\xi}d\xi.$$
现只需证明, 任取$\varepsilon>0$, 存在$h>0$使得$|F[f](\lambda+h)-F[f](\lambda)|\leqslant\varepsilon$. 将上式拆分为三个区间的积分
$$|F[f](\lambda+h)-F[f](\lambda)|\leqslant\sqrt{2}\left(\int_{-\infty}^{A}|f(\xi)|\sqrt{1-\cos h\xi}d\xi+\int_{A}^{B}|f(\xi)|\sqrt{1-\cos h\xi}d\xi+\int_{B}^{\infty}|f(\xi)|\sqrt{1-\cos h\xi}d\xi\right).$$
由于$f(x)$在$(-\infty,\infty)$上绝对可积, 可找到$A<0$, $B>0$使得
$$\sqrt{2}\int_{-\infty}^{A}|f(\xi)|\sqrt{1-\cos h\xi}d\xi\leqslant\sqrt{2}\int_{-\infty}^{A}|f(\xi)|d\xi\leqslant\frac{\varepsilon}{3},$$
$$\sqrt{2}\int_{B}^{\infty}|f(\xi)|\sqrt{1-\cos h\xi}d\xi\leqslant\sqrt{2}\int_{B}^{\infty}|f(\xi)|d\xi\leqslant\frac{\varepsilon}{3}.$$
设
$$C=\max\{-A,B\},\quad M=\sup\limits_{x\in[A,B]}|f(x)|,\quad L=B-A,$$
$$\sqrt{2}\int_{A}^{B}|f(\xi)|\sqrt{1-\cos h\xi}d\xi\leqslant\sqrt{2}(B-A)\sup\limits_{x\in[A,B]}|f(x)|\sup\limits_{x\in[A,B]}\sqrt{1-\cos hx}=\sqrt{2}ML\sqrt{1-\cos hC}.$$
令$$h\leqslant\frac{1}{C}\arccos\left(1-\frac{\varepsilon^2}{18M^2L^2}\right),$$
则
$$\sqrt{2}\int_{A}^{B}|f(\xi)|\sqrt{1-\cos h\xi}d\xi\leqslant\frac{\varepsilon}{3},$$
$$|F[f](\lambda+h)-F[f](\lambda)|\leqslant\varepsilon.$$
故得证.

\end{document}
