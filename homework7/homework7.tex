\documentclass[11pt,a4paper]{article}
\usepackage{../ma319}
\semester{Fall}
\year{2019}
\subtitlenumber{7}
\author{刘逸灏 (515370910207)}

\begin{document}

\maketitle

\section{2.2/3}
\begin{problem}
如果有一长度为$l$的均匀细棒, 其周围及两端$x=0$, $x=l$均为绝热, 初始温度分布为$u(x,0)=f(x)$, 问以后时刻的温度分布如何? 且证明当$f(x)$等于常数$u_0$时, 恒有$u(x,t)=u_0$.
\end{problem}

题目可归结为求解如下的定解问题:
$$\left\{\begin{aligned}
     & \frac{\partial u}{\partial t}=a^2\frac{\partial^2u}{\partial x^2},        \\
     & \frac{\partial u}{\partial x}(0,t) =\frac{\partial u}{\partial x}(l,t)=0, \\
     & u(x,0)=f(x).
  \end{aligned}\right.$$

方程的特征值和对应的特征函数为
$$X''(x)+\lambda X(x)=0,$$
$$X(x)=C_1\cos\sqrt{\lambda}x+C_2\sin\sqrt{\lambda}x,$$

代入$X'(0)=0$, $X'(l)=0$得
$$C_2=0,\quad C_1\sqrt{\lambda}\sin\sqrt{\lambda}l=0,$$
$$\lambda=\lambda_k=\frac{k^2\pi^2}{l^2},\quad X_k(x)=C_k\cos\frac{k\pi}{l}x,\quad k=0,1,2,\cdots.$$

方程的通解为
$$u(x,t)=\frac{1}{2}A_0+\sum_{k=1}^\infty A_ke^{-a^2\lambda t}\cos\sqrt{\lambda}x=\frac{1}{2}A_0+\sum_{k=1}^\infty A_k e^{-\frac{k^2\pi^2}{l^2}a^2 t}\cos\frac{k\pi}{l}x.$$

代入初值条件得
$$u(x,0)=\frac{1}{2}A_0+\sum_{k=1}^\infty A_k\cos\sqrt{\lambda}x=\frac{1}{2}A_0+\sum_{k=1}^\infty A_k\cos\frac{k\pi}{l}x=f(x).$$

解得
$$A_k=\frac{2}{l}\int_0^l f(x)\cos\sqrt{\lambda}xdx=\frac{2}{l}\int_0^l f(x)\cos\frac{k\pi}{l}xdx.$$

故
$$u(x,t)=\frac{1}{l}\int_0^l f(x)dx+\sum_{k=1}^\infty \frac{2}{l}\int_0^l f(x)\cos\frac{k\pi}{l}xdx \cdot e^{-\frac{k^2\pi^2}{l^2}a^2 t}\cos\frac{k\pi}{l}x.$$

当$f(x)$等于常数$u_0$时
$$A_k=\left\{\begin{aligned}
     & 2u_0,\quad                         & k=0, \\
     & \frac{u_0l\sin k\pi}{k\pi}=0,\quad & k>0.
  \end{aligned}\right.$$

故
$$u(x,t)=\frac{1}{2}A_0=u_0.$$

\section{2.2/4}
\begin{problem}
在区域$t>0$, $0<x<l$中求解如下的定解问题:
$$\left\{\begin{aligned}
     & \frac{\partial u}{\partial t}=\frac{\partial^2u}{\partial x^2}-\beta(u-u_0), \\
     & u(0,t) = u(1,t)=u_0,                                                         \\
     & u(x,0)=f(x),
  \end{aligned}\right.$$
其中$a$, $\beta$, $u_0$均为常数, $f(x)$为已知函数.
\end{problem}

设
$$u(x,t)=u_0+v(x,t)e^{-\beta t}.$$

则
$$\frac{\partial u}{\partial t}-\frac{\partial^2u}{\partial x^2}+\beta(u-u_0)=\frac{\partial v}{\partial t}e^{-\beta t}-\beta ve^{-\beta t}-\frac{\partial^2v}{\partial x^2}e^{-\beta t}+\beta ve^{-\beta t}=e^{-\beta t}\left(\frac{\partial v}{\partial t}-\frac{\partial^2v}{\partial x^2}\right)=0,$$
$$v(0,t)e^{-\beta t}=v(l,t)e^{-\beta t}=u(0,t)-u_0=u(l,t)-u_0=0,$$
$$v(x,0)=u(x,0)-u_0=f(x)-u_0.$$

故可以先求解如下定解问题:
$$\left\{\begin{aligned}
     & \frac{\partial v}{\partial t}=\frac{\partial^2v}{\partial x^2}, \\
     & v(0,t) = v(1,t)=0,                                              \\
     & v(x,0)=f(x)-u_0.
  \end{aligned}\right.$$

方程的特征值和对应的特征函数为
$$\lambda=\lambda_k=\frac{k^2\pi^2}{l^2},\quad X_k(x)=C_k\sin\sqrt{\lambda}x=C_k\sin\frac{k\pi}{l} x,\quad k=1,2,\cdots.$$

方程的通解为
$$v(x,t)=\sum_{k=1}^\infty A_ke^{-a^2\lambda t}\sin\sqrt{\lambda}x=\sum_{k=1}^\infty A_ke^{-\frac{k^2\pi^2}{l^2}a^2t}\sin \frac{k\pi}{l} x.$$

代入初值条件得
$$v(x,0)=\sum_{k=1}^\infty A_k\sin\sqrt{\lambda}x=\sum_{k=1}^\infty A_k\sin \frac{k\pi}{l} x=f(x)-u_0.$$

解得
$$A_k=\frac{2}{l}\int_0^l [f(x)-u_0]\sin\sqrt{\lambda}xdx=\frac{2}{l}\int_0^l [f(x)-u_0]\sin \frac{k\pi}{l} xdx
  =\frac{2}{l}\int_0^l f(x)\sin \frac{k\pi}{l} xdx-\frac{2u_0(1-\cos k\pi)}{k\pi}.$$

化简得
$$A_k=\frac{2}{l}\int_0^l f(x)\sin \frac{k\pi}{l} xdx+\frac{2u_0[(-1)^k-1]}{k\pi}.$$

故
$$v(x,t)=\sum_{k=1}^\infty \left\{\frac{2}{l}\int_0^l f(x)\sin \frac{k\pi}{l} xdx+\frac{2u_0[(-1)^k-1]}{k\pi}\right\} e^{-\frac{k^2\pi^2}{l^2}a^2t}\sin \frac{k\pi}{l} x,$$
$$u(x,t)=u_0+e^{-\beta t}\sum_{k=1}^\infty \left\{\frac{2}{l}\int_0^l f(x)\sin \frac{k\pi}{l} xdx+\frac{2u_0[(-1)^k-1]}{k\pi}\right\} e^{-\frac{k^2\pi^2}{l^2}a^2t}\sin \frac{k\pi}{l} x.$$

\section{2.2/5}
\begin{problem}
长度为$l$的均匀细杆的初始温度为$0^\circ C$, 端点$x=0$保持常温$u_0$, 而在$x=l$和侧面上, 热量可以发散到周围的介质中去, 介质的温度为$^\circ C$, 此时杆上的温度分布函数$u(x,t)$满足下述定解问题:
$$\left\{\begin{aligned}
     & \frac{\partial u}{\partial t}=a^2\frac{\partial^2u}{\partial x^2}-b^2u,              \\
     & u(0,t)=u_0,\quad \left.\left(\frac{\partial u}{\partial x}+Hu\right)\right|_{x=l}=0, \\
     & u(x,0)=0,
  \end{aligned}\right.$$
其中$a,b,H$均为常数, 试求出$u(x,t)$.
\end{problem}

设
$$u(x,t)=v(x,t)e^{-b^2t}+f(x).$$

则
$$\frac{\partial u}{\partial t}-a^2\frac{\partial^2u}{\partial x^2}+b^2u=\frac{\partial v}{\partial t}e^{-b^2t}-b^2ve^{-b^2t}-a^2\frac{\partial^2v}{\partial x^2}e^{-b^2t}-a^2f''(x)+b^2ve^{-b^2t}+b^2f(x)=0,$$
$$v(0,t)e^{-b^2t}+f(0)=u_0,\quad \left.\left(\frac{\partial v}{\partial x}+Hv\right)\right|_{x=l}+f'(l)+Hf(l)=0,$$
$$v(x,0)+f(x)=0.$$

可找到$f(x)$使以下成立
$$\left(\frac{\partial v}{\partial t}-\frac{\partial^2v}{\partial x^2}\right)e^{-b^2t}=0,\quad f''(x)-\frac{b^2}{a^2}f(x)=0,$$
$$v(0,t)e^{-b^2t}=0,\quad f(0)=u_0,\quad \left.\left(\frac{\partial v}{\partial x}+Hv\right)\right|_{x=l}=0,\quad f'(l)+Hf(l)=0.$$

故$f(x)$的通解为
$$f(x)=C_1e^{\frac{b}{a}x}+C_2e^{-\frac{b}{a}x}.$$
代入初值条件得
$$f(0)=C_1+C_2=u_0,$$
$$f'(l)+Hf(l)=\frac{b}{a}\left(C_1e^{\frac{bl}{a}}-C_2e^{-\frac{bl}{a}}\right)+H\left(C_1e^{\frac{bl}{a}}+C_2e^{-\frac{bl}{a}}\right)=0.$$
解得
$$$$

\section{2.3/1}

\subsection*{(1)}
\subsection*{(2)}

\section{2.3/2}




\end{document}
